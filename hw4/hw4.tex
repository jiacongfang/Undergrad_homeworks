\documentclass[11pt]{article}           
\usepackage[UTF8]{ctex}
\usepackage[a4paper]{geometry}
\geometry{left=2.0cm,right=2.0cm,top=2.5cm,bottom=2.5cm}

\usepackage{xcolor}
\usepackage{paralist}
\usepackage{enumitem}
\setenumerate[1]{itemsep=0pt,partopsep=0pt,parsep=0pt,topsep=0pt}
\setitemize[1]{itemsep=0pt,partopsep=0pt,parsep=0pt,topsep=0pt}
\usepackage{comment}
\usepackage{booktabs}
\usepackage{graphicx}
\usepackage{float}
\usepackage{diagbox}
\usepackage{amsmath,amsfonts,graphicx,amssymb,bm,amsthm}
%\usepackage{algorithm,algorithmicx}
\usepackage[ruled]{algorithm2e}
\usepackage[noend]{algpseudocode}
\usepackage{fancyhdr}
\usepackage{tikz}
\usepackage{graphicx}
\usetikzlibrary{arrows,automata}
\usepackage[hidelinks]{hyperref}
\usepackage{extarrows}
\usepackage{lastpage}
\usepackage{totcount}
\setlength{\headheight}{14pt}
\setlength{\parindent}{0 in}
\setlength{\parskip}{0.5 em}
\usepackage{helvet}
\usepackage{dsfont}
\usepackage{threeparttable}
\usepackage{multirow}
\usepackage{tabularx}
\usepackage{makecell}
\usepackage{caption}
% \usepackage{newtxmath}

\newtheorem{theorem}{Theorem}
\newtheorem{lemma}[theorem]{Lemma}
\newtheorem{proposition}[theorem]{Proposition}
\newtheorem{claim}[theorem]{Claim}
\newtheorem{corollary}[theorem]{Corollary}
\newtheorem{definition}[theorem]{Definition}
\newtheorem*{definition*}{Definition}

\newenvironment{problem}[2][Problem]{\begin{trivlist}
\item[\hskip \labelsep {\bfseries #1}\hskip \labelsep {\bfseries #2.}]\songti}{\hfill$\blacktriangleleft$\end{trivlist}}
\newenvironment{answer}[1][Solution]{\begin{trivlist}
\item[\hskip \labelsep {\bfseries #1.}\hskip \labelsep]}{\hfill$\lhd$\end{trivlist}}

\newcommand\1{\mathds{1}}
% \newcommand\1{\mathbf{1}}
\newcommand\R{\mathbb{R}}
\newcommand\E{\mathbb{E}}
\newcommand\N{\mathbb{N}}
\newcommand\NN{\mathcal{N}}
\newcommand\per{\mathrm{per}}
\newcommand\PP{\mathbb{P}}
\newcommand\dd{\mathrm{d}}
\newcommand\Var{\mathrm{Var}}
\newcommand\Cov{\mathrm{Cov}}
\newcommand{\Exp}{\mathrm{Exp}}
\newcommand{\arrp}{\xrightarrow{P}}
\newcommand{\arrd}{\xrightarrow{d}}
\newcommand{\arras}{\xrightarrow{a.s.}}
\newcommand{\arri}{\xrightarrow{n\rightarrow\infty}}

\renewcommand{\arraystretch}{0.9}


\title{Homework \#3}
\usetikzlibrary{positioning}

\begin{document}
\kaishu

\pagestyle{fancy}
\lhead{\CJKfamily{zhkai} 北京大学}
\chead{}
\rhead{\CJKfamily{zhkai} 2024年秋\ 社会统计学(张春泥)}
\fancyfoot[C]{\thepage\ /\ \pageref{LastPage} \\ \textcolor{lightgray}{最后编译时间: \today}}



\begin{center}
    {\LARGE \bf Homework 4}

    {姓名:方嘉聪\ \  学号: 2200017849}            % Write down your name and ID here.
\end{center}
\begin{problem}{1}
    \begin{enumerate}[label=(\arabic*)]
        \item 变量分布见表\ref{tab:1.1}:
    \begin{table}[H]
        \centering
        \caption{相关变量的均值及标准差分布($N=15862$)}
        \label{tab:1.1}
        \begin{tabularx}{0.9\textwidth}{l>{\centering\arraybackslash}X>{\centering\arraybackslash}X}
            \hline
            \textbf{变量} & \textbf{均值} & \textbf{方差}  \\
            \hline
            居民收入对数 & 7.44 & 0.43 \\
            性别(男=0)   & 0.48 & 0.50 \\
            教育年限 & 10.67 & 3.12 \\
            工作年限 & 19.72 & 10.25\\
            党员身份(非党员=0) & 0.24 & 0.43 \\
            \hline
        \end{tabularx}
    \end{table} 
    \item 回归分析结果见表\ref{tab:1.2}:
    \begin{table}[H]
        \centering 
        \caption{居民收入影响因素的多元线性回归模型回归表($N=15862$)}
        \label{tab:1.2}
        \begin{tabularx}{0.9\textwidth}{l>{\centering\arraybackslash}X>{\centering\arraybackslash}X>{\centering\arraybackslash}X>{\centering\arraybackslash}X}
            \hline
            \textbf{变量} & \textbf{模型1} & \textbf{模型2} & \textbf{模型3} & \textbf{模型4}  \\
            \hline
            \multirow{2}{*}{性别($\text{男}=0$)} & $-0.163^{**}$ & $-0.125^{**}$ & $-0.114^{**}$ & $-0.113^{**}$\\
             & (.006) & (.006) & (.006) & (.006)\\
            \multirow{2}{*}{工作年限}   & $0.044^{**}$ & $0.046^{**}$ & $0.044^{**}$ & $0.046^{**}$ \\
                & (.001) & (.001) & (.001) & (.001)\\
            \multirow{2}{*}{工作年限的平方项} &$-0.001^{**}$ & $-0.001^{**}$ & $-0.001^{**}$ & $-0.001^{**}$ \\
                & (.000) & (.000) & (.000) & (.000)\\
            \multirow{2}{*}{教育年限} & \multirow{2}{*}{/} & $0.035^{**}$ & $0.031^{**}$ & $0.036^{**}$ \\
             &  & (.001) & (.001) & (.001)\\
            \multirow{2}{*}{党员身份($\text{非党员}=0$)} & \multirow{2}{*}{/} & \multirow{2}{*}{/} & $0.071^{**}$ & $0.230^{**}$ \\
             &  &  & (.008) & (.026)\\
            \multirow{2}{*}{教育年限 $\times$ 党员身份} & \multirow{2}{*}{/} & \multirow{2}{*}{/} & \multirow{2}{*}{/} & $-0.014^{**}$ \\
             &  &  &  & (.002)\\ \\
            \multirow{2}{*}{截距项} & $7.001^{**}$ & $6.558^{**}$ & $6.591^{**}$ & $6.538^{**}$ \\
             & (.011) & (.017) & (.017) & (.019)\\
            $R^2$ & 0.2045 & 0.2575 & 0.2614 & 0.2633 \\
            \hline
        \end{tabularx}
        \begin{tablenotes}
            \footnotesize
            \item 注:$*: p<0.05, **:p<0.01$. 括号中的数字为标准误. / 表示模型不包含该变量. (教育年限 $\times$ 党员身份)表示教育年限与党员身份的交互项.
        \end{tablenotes}
    \end{table}
    \item 为检验教育年限和党员身份对因变量的影响的联合效应, 考虑使用以下两个模型:
    \begin{align*}
        \text{限制性模型: 模型1} \quad &\text{logearn} = \beta_0 + \beta_1 \text{sex} + \beta_2 \text{exp} + \beta_3 \text{exp}^2 + \varepsilon, \\
        \text{非限制性模型: 模型3} \quad &\text{logearn} = \beta_0 + \beta_1 \text{sex} + \beta_2 \text{exp} + \beta_3 \text{exp}^2 + \beta_4 \text{edu} + \beta_5 \text{cpc} + \varepsilon.
    \end{align*}
    考虑显著性水平$\alpha=0.01$下, 原假设与备择假设分别为:
    \begin{align*}
        H_0: \beta_4 = \beta_5 = 0, \quad H_1: \beta_4 \neq 0 \lor \beta_5 \neq 0.
    \end{align*} 
    $F$ 值为 
    \begin{align*}
        F = \frac{(\text{SSE}_R - \text{SSE}_U)/q}{\text{SSE}_U/(n-K)} = \frac{(2347.24-2179.22)/2}{2179.22/(15862 - 6)} = 611.26 > F_{0.01}(2, 15856) = 4.61.
    \end{align*}
    故拒绝原假设. 即在显著性水平$\alpha=0.01$下, 教育年限和党员身份对因变量的影响具有显著联合效应.
    \item  斜率系数$\beta_1$表示取值等于1的类别(虚拟变量)与参照组之间在因变量上的均值差, 若取值对调, 则差值的符号会变为相反数. 

    在模型1中将男女取值对调后, 性别对应的变量估计的斜率系数会变为原有的相反数, 同时截距项变为新参照组的截距. 具体地, 模型1可改写为
    \[
    \text{logearn} = \beta_0' + \beta_1' \text{sex} + \beta_2 \text{exp} + \beta_3 \text{exp}^2 + \varepsilon, \text{ 其中 } \beta_1' = -\beta_1, \beta_0' = \beta_0 + \beta_1.
    \]
    若带入回归结果, 则有
    \[
    \text{logearn} = 6.838 + 0.163 \text{sex} +  0.044 \text{exp} - 0.001 \text{exp}^2 + \varepsilon.
    \]
    \item 根据回归模型(即模型4)的估计结果, 解释如下:
    \begin{itemize}
        \item {\kaishu 教育年限变量的主效应($\beta_4 = 0.036$):} 在控制其他变量不变的情况下, 教育年限每增加一个单位, 居民收入的对数期望值将增加$0.036$个单位. 说明教育年限对居民收入有 \underline{正向}影响.
        \item {\kaishu 党员身份虚拟变量(参照组 $\text{非党员}=0$, $\beta_5 = 0.230$)的主效应:} 在控制其他变量不变的情况下, 党员收入的对数期望值相较于非党员将增加$0.230$个单位. 说明党员身份对居民收入有 \underline{正向}影响.
        \item {\kaishu 教育年限与党员身份的交互效应:} 偏回归系数 $\beta_6 = -0.014$刻画了这两个变量对因变量的非线性作用, 表明教育年限对收入的作用和党员身份对收入的作用间存在着 \underline{相互削弱} 的关系。
    \end{itemize}
\end{enumerate}

\end{problem}
\end{document}