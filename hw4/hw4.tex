\documentclass[11pt]{article}           
\usepackage[UTF8]{ctex}
\usepackage[a4paper]{geometry}
\geometry{left=2.0cm,right=2.0cm,top=2.5cm,bottom=2.25cm}

\usepackage{xcolor}
\usepackage{paralist}
\usepackage{enumitem}
\setenumerate[1]{itemsep=0pt,partopsep=0pt,parsep=0pt,topsep=0pt}
\setitemize[1]{itemsep=0pt,partopsep=0pt,parsep=0pt,topsep=0pt}
\usepackage{comment}
\usepackage{booktabs}
\usepackage{graphicx}
\usepackage{float}
\usepackage{sgame} % For Game Theory Matrices 
% \usepackage{diagbox} % Conflict with sgame
\usepackage{amsmath,amsfonts,graphicx,amssymb,bm,amsthm}
%\usepackage{algorithm,algorithmicx}
\usepackage[ruled]{algorithm2e}
\usepackage[noend]{algpseudocode}
\usepackage{fancyhdr}
\usepackage{tikz}
\usepackage{pgfplots}
\pgfplotsset{compat=1.18}
\usepackage{graphicx}
\usetikzlibrary{arrows,automata}
\usepackage[hidelinks]{hyperref}
\usepackage{extarrows}
\usepackage{totcount}
\setlength{\headheight}{14pt}
\setlength{\parindent}{0 in}
\setlength{\parskip}{0.5 em}
\usepackage{helvet}
\usepackage{dsfont}
% \usepackage{newtxmath}
\usepackage[labelfont=bf]{caption}
\renewcommand{\figurename}{Figure}
\usepackage{lastpage}
\usepackage{istgame}
\usepackage{tcolorbox}
% \newdateformat{mydate}{\shortmonthname[\THEMONTH]. \THEDAY \THEYEAR}

\newtheorem{theorem}{Theorem}
\newtheorem{lemma}[theorem]{Lemma}
\newtheorem{proposition}[theorem]{Proposition}
\newtheorem{claim}[theorem]{Claim}
\newtheorem{corollary}[theorem]{Corollary}
\newtheorem{definition}[theorem]{Definition}
\newtheorem*{definition*}{Definition}

\newenvironment{problem}[2][Problem]{\begin{trivlist}
    \item[\hskip \labelsep {\bfseries #1}\hskip \labelsep {\bfseries #2.}]\songti}{\hfill$\blacktriangleleft$\end{trivlist}}
\newenvironment{answer}[1][Solution]{\begin{trivlist}
    \item[\hskip \labelsep {\bfseries #1.}\hskip \labelsep]}{\hfill$\lhd$\end{trivlist}}

\newcommand\1{\mathds{1}}
% \newcommand\1{\mathbf{1}}
\newcommand\R{\mathbb{R}}
\newcommand\E{\mathbb{E}}
\newcommand\N{\mathbb{N}}
\newcommand\NN{\mathcal{N}}
\newcommand\per{\mathrm{per}}
\newcommand\PP{\mathbb{P}}
\newcommand\dd{\mathrm{d}}
\newcommand\Var{\mathrm{Var}}
\newcommand\Cov{\mathrm{Cov}}
\newcommand{\Exp}{\mathrm{Exp}}
\newcommand{\arrp}{\xrightarrow{P}}
\newcommand{\arrd}{\xrightarrow{d}}
\newcommand{\arras}{\xrightarrow{a.s.}}
\newcommand{\arri}{\xrightarrow{n\rightarrow\infty}}
\newcommand{\iid}{\overset{\text{i.i.d}}{\sim}}

\definecolor{lightgray}{gray}{0.75}

\DeclareMathOperator*{\argmax}{argmax} % 定义 \argmax 运算符
\title{Homework \#3}
\usetikzlibrary{positioning}

\begin{document}
\kaishu

\pagestyle{fancy}
\lhead{\CJKfamily{zhkai} 北京大学}
\chead{}
\rhead{\CJKfamily{zhkai} 2024年秋\ 信息学中的概率统计(王若松)}
\fancyfoot[R]{} 
\fancyfoot[C]{\thepage\ /\ \pageref{LastPage} \\ \textcolor{lightgray}{最后编译时间: \today}}


\begin{center}
    {\LARGE \bf Homework 4}

    {姓名:方嘉聪\ \  学号: 2200017849}            % Write down your name and ID here.
\end{center}

\begin{problem}{1}
    一个盒子中有 $n$ 个小球, 编号分别为 $1,2, \cdots, n$. 从盒子中取出 $k\le n$ 个小球, 每次等概率从盒子中剩余的小球中取出一个, 且每次取完后均不放回. 也即, 第一次取小球时, 每个小球被取出的概率均为 $1/n$. 第二次取小球时, 剩余的 $n-1$ 个小球各自被取出的概率均为 $1/(n-1)$. 以此类推, 直至一共取出 $k$ 个小球. 
    
    定义随机变量 $X_1, X_2, \cdots, X_k$, 其中 $X_i (1\le i\le k)$ 表示第 $i$ 次取出小球的编号.
    \begin{enumerate}[label=(\arabic*)]
        \item 对于 $1\le i < j \le k$, 判断 $X_i$ 是否与 $X_j$ 相互独立.
        \item 计算 $X_1, X_2, \cdots, X_k$ 的联合分布列.
        \item 对于 $1\le i\le k$, 计算 $X_i$ 的边际分布列.
        \item 对于任意 $1\le i < j \le k$ 和 $1\le a_i,a_j \le n$, 计算 $\PP(X_i = a_i \cap X_j = a_j)$.
        \item 利用恒等式 \begin{align*}
            \sum_{i=1}^{n} i^2 = \frac{n(n+1)(2n+1)}{6}
        \end{align*}
        对于 $1\le i\le k$, 计算 $\E(X_i)$ 和 $\Var(X_i)$.
        \item 对于 $1\le i < j \le k$, 计算 $\Cov(X_i, X_j)$.
    \end{enumerate}
\end{problem}
\begin{answer}
    \begin{enumerate}[label=(\arabic*)]
        \item 注意到 对于任意 $x_i, x_j \in \{1,\cdots, n\}$, 相当于 第$i/j$ 位已经确定, 余下排列即可. 故有
        \begin{align*}
            \PP(X_i = x_i) = \frac{(n-1)!}{n!} = \frac{1}{n}, \text{ 同理 }\quad  \PP(X_j = x_j) = \frac{1}{n}.
        \end{align*}
        而当 $x_i = x_j$ 时, $\PP(X_i = x_i, X_j = x_j) = 0 \neq \PP(X_i = x_i) \cdot \PP(X_j = x_j)$, 故 $X_i$ 与 $X_j$ 不相互独立. 
        \item 对于 $\{x_i\}_{i=1}^k \in \{1,2, \cdots, n\}$, 若 $\exists i\neq j$ 使得 $x_i = x_j$, 则 $\PP(X_1 = x_1, X_2 = x_2, \cdots, X_k = x_k) = 0$. 否则, 有
        \begin{align*}
            \PP(X_1 = x_1, X_2 = x_2, \cdots, X_k = x_k) = \prod_{i=1}^{k} \frac{1}{n-i+1} = \frac{(n-k)!}{n!}.
        \end{align*} 
        \item 对于 $1\le i\le k$, 由 (1) 可知 $\PP(X_i = x_i) = 1/n, \forall x_i \in \{1,2, \cdots, n\}$. 否则, 有 $\PP(X_i = x_i) = 0$.
        \item 若 $a_i = a_j$, 则 $\PP(X_i = a_i \cap X_j = a_j) = 0$. 否则, 有
        \begin{align*}
            \PP(X_i = a_i \cap X_j = a_j) = \frac{(n-2)!}{n!} = \frac{1}{n(n-1)}.
        \end{align*}
        综上,
        \begin{align*}
            \PP(X_i = a_i \cap X_j = a_j) = \begin{cases}
                0, & a_i = a_j, \\
                \frac{1}{n(n-1)}, & a_i \neq a_j.
            \end{cases}
        \end{align*}
        \item 先计算 $\E(X_i)$, 有
        \begin{align*}
            \E(X_i) = \sum_{x_i=1}^{n} x_i \cdot \PP(X_i = x_i) = \frac{1}{n} \sum_{x_i=1}^{n} x_i = \frac{n+1}{2}.
        \end{align*}
        再计算 $\E(X_i^2)$, 有
        \begin{align*}
            \E(X_i^2) = \sum_{x_i=1}^{n} x_i^2 \cdot \PP(X_i = x_i) = \frac{1}{n} \sum_{x_i=1}^{n} x_i^2 = \frac{(n+1)(2n+1)}{6}
        \end{align*}
        那么
        \begin{align*}
            \Var(X_i) = \E(X_i^2) - \E(X_i)^2 = \frac{n^2-1}{12}.
        \end{align*}
        \item 我们来计算 $\E(X_iX_j)$, 其中 $1 \le i<j \le k$, 
        \begin{align*}
            \E(X_iX_j) = \sum_{a_i=1}^{n} \sum_{a_j=1}^{n} a_i a_j \PP(X_i = a_i, X_j = a_j)  = \frac{1}{n(n-1)} \sum_{a_i \neq a_j} a_i a_j = \frac{(n+1)(3n+2)}{12} 
        \end{align*}
        那么
        \begin{align*}
            \Cov(X_i, X_j) = \E(X_iX_j) - \E(X_i)\E(X_j) = -\frac{n+1}{12}
        \end{align*}
    \end{enumerate}
\end{answer}

\begin{problem}{2}
    将 $n$ 个编号为 $1,2, \cdots, n$ 的小球随机打乱, 每一种排列等概率出现. 用 $\pi_1, \cdots, \pi_n$ 表示随机打乱后每个位置上的小球编号, 也即 $\pi_i$ 表示随机打乱后, 位置为 $i$ 的小球的原始编号. 对于 $1< i < n$, 我们称 $i$ 是一个 {\kaishu 局部极大值}, 当且仅当 $\pi_{i} > \pi_{i-1}$ 且 $\pi_{i} > \pi_{i+1}$.  
    令随机变量 $X$ 表示所有 $1<i<n$ 中局部极大值的总数量. 计算 $\E(X)$.
\end{problem}
\begin{answer}
    记随机变量 $X_i = \1_{i\text{为局部极大值}}$, 那么 $X = \sum_{i=2}^{n-1} X_i$. 而
    \begin{align*}
        \E(X_i) = \PP(i\text{为局部极大值}) = \frac{\sum_{i=3}^{n}\binom{i-1}{2}\cdot 2! \cdot (n-3)!}{n!} = \frac{\sum_{i=1}^{n}(i-1)(i-2)}{n(n-1)(n-2)} = \frac{1}{3}.
    \end{align*}
    那么
    \begin{align*}
        \E(X) = \sum_{i=2}^{n-1}\E(X_i) = \frac{n-2}{3}.
    \end{align*}
\end{answer}

\begin{problem}{3}
    令随机变量 $X \sim G(p)$, 即随机变量 $X$ 服从参数为 $p$ 的几何分布. 证明
    \begin{align*}
        \E\left(X^2\right) = p + \E\left((X+1)^2\right) (1-p), \\
        \E\left(X^3\right) = p + \E\left((X+1)^3\right) (1-p).
    \end{align*}
    并计算 $\E\left(X^2\right)$ 和 $\E\left(X^3\right)$.
\end{problem}
\begin{answer}
    \begin{enumerate}[label=(\arabic*)]
        \item 按照定义展开
        \begin{align*}
            \E\left(X^2\right) &= 1^2 \cdot \PP(X=1) + \sum_{k=2}^{+\infty} k^2 \cdot \PP(X=k) = p + \PP(X > 1) \sum_{k=2}^{+\infty} k^2 \cdot \PP(X=k|X>1) \\
            &= 1 + (1-p) \sum_{k=2}^{+\infty} k^2 \cdot \frac{p(1-p)^{k-1} }{1-p} = 1 + (1-p) \sum_{k=2}^{+\infty} k^2 \cdot \PP(X = k-1) \\
            &= 1 + (1-p) \sum_{k=1}^{+\infty} (k+1)^2 \cdot \PP(X = k) = p + \E\left((X+1)^2\right) (1-p).
        \end{align*}
        对于 $\E\left(X^3\right)$, 同理可得 $\E\left(X^3\right) = p + \E\left((X+1)^3\right) (1-p)$.

        那么($\E(X) = 1/p$)
        \begin{align*}
            \E\left(X^2\right) = p + (1-p) \cdot \left(\E\left(X^2\right) + 2\E(X) + 1  \right) \implies \E\left(X^2\right) = \frac{2-p}{p^2}.
        \end{align*}
        同理可得 
        \begin{align*}
            \E\left(X^3\right) = p + (1-p)\left(\E(X^3) + 3\E(X^2) + 3\E(X) + 1\right) \implies \E\left(X^3\right) = \frac{p^2-6p+6}{p^3}.
        \end{align*}
    \end{enumerate}
\end{answer}

\begin{problem}{4}
    令 $X_1, X_2, \cdots$ 为一列同分布的离散随机变量. 离散随机变量 $N$ 取正整数值, 且 $N, X_1,$ $X_2, \cdots,$ 相互独立. 在课上我们证明了 
    \begin{align*}
        \E\left(\sum_{i=1}^{N}X_i\right) = \E(N)\cdot\E(X_1).
    \end{align*} 
    回答下列问题:
    \begin{enumerate}[label=(\arabic*)]
        \item 给出例子使得 
        \begin{align*}
            \Var\left(\sum_{i=1}^{N} X_i\right) \neq \E(N) \cdot \Var(X_1).
        \end{align*}
        \item 证明
        \begin{align*}
            \Var\left(\sum_{i=1}^{N} X_i\right) = \E(N)\cdot \Var(X_1) + \Var(N)\cdot \E(X_1)^2.
        \end{align*}
    \end{enumerate}
\end{problem}
\begin{answer}
    \begin{enumerate}[label=(\arabic*)]
        \item 任意 $i\in \N^+$, 令$X_i$ 同分布于单点分布, 即 $\PP(X_i = c) = 1$, 其中 $c\in \R_{\neq 0}$ 为 给定常数. 那么 $\Var(X_i) = 0$. 令 $N$ 服从两点分布, 特别地令 $\PP(N = 1) = \PP(N = 2) = 1/2$. 
        那么
        \begin{align*}
            \PP\left(\sum_{i=1}^{N}X_i = c\right) = \frac{1}{2}, \quad \PP\left(\sum_{i=1}^{N}X_i = 2c\right) = \frac{1}{2}.
        \end{align*}
        进而
        \begin{align*}
            \Var\left(\sum_{i=1}^{N}X_i\right) = \frac{1}{2} \cdot \frac{2c^2}{4} = \frac{c^2}{4}  \neq 0 = \E(N) \cdot \Var(X_1).
        \end{align*}
        \item 由课上我们证明的结论, 可知
        \begin{align*}
            \E\left[\left(\sum_{i=1}^{N} X_i\right)^2\right] &= \E\left[\sum_{i=1}^{N}X_i^2 + 2 \sum_{1\le i < j \le N}X_i X_j\right] = \E\left[\sum_{i=1}^{N}X_i^2\right] + \E\left[\sum_{1\le i < j \le N}2X_i X_j\right] \\
            &= \E(N) \cdot \E(X_1^2) + \E(N^2 - N) \cdot \E(X_1)^2  =  \E(N)\cdot \Var(X_1) + \E(N^2)\cdot E(X_1)^2
        \end{align*}
        那么
        \begin{align*}
            \Var\left(\sum_{i=1}^{N}X_i\right) &= \E\left[\left(\sum_{i=1}^{N}X_i\right)^2\right] - \E\left[\sum_{i=1}^{N}X_i\right]^2 \\
            &= \E(N)\Var(X_1) + \E(N^2)E(X_1)^2 - \E(N)^2\E(X_1)^2 \\
            &= \E(N)\cdot \Var(X_1) + \Var(N)\cdot \E(X_1)^2.
        \end{align*}
        证毕.
    \end{enumerate}
\end{answer}

\begin{problem}{5}
    \begin{enumerate}[label=(\arabic*)]
        \item 对于正整数 $r$ 和实数 $p\in (0,1)$, 给定 $X\sim NB(1,p), Y\sim NB(r,p)$. 若 $X$ 和 $Y$ 相互独立, 证明 $X+Y\sim NB(r+1,p)$. 
        {\kaishu 提示: 使用恒等式}
        \begin{align*}
            \binom{n}{m} = \binom{n-1}{m-1} + \binom{n-1}{m}.
        \end{align*}
        \item 对于正整数 $r$, 随机变量$X_1, \cdots, X_r \iid G(p)$, 证明 $\sum_{i=1}^{r}X_i \sim NB(r,p)$.
    \end{enumerate}
\end{problem}
\begin{answer}
\begin{enumerate}[label = (\arabic*)]
    \item 首先证明一个组合恒等式
    \begin{align*}
        \sum_{k=1}^{n-1} \binom{n-k-1}{r-1} = \sum_{k=1}^{n-1} \binom{n-k}{r} - \binom{n-k-1}{r} =\binom{n-1}{r}.
    \end{align*}  
    又注意到 $X,Y \in \N^+$, 那么对于任意 $n \in \N_{\ge 2}$: 
    \begin{align*}
        \PP(X + Y = n) &= \sum_{k=1}^{n} \PP(X+Y = n | X = k) \PP(X = k) \\
        &= \sum_{k=1}^{n-1} \PP(Y = n-k) \PP(X=k) = \sum_{k=1}^{n-1} p(1-p)^{k-1} \cdot \binom{n-k-1}{r-1} p^r(1-p)^{n-k-r} \\
        &= p^{r+1} (1-p)^{n-(r+1)} \sum_{k=1}^{n-1} \binom{n-k-1}{r-1} \\
        &= \binom{n-1}{r} p^{r+1} (1-p)^{n-(r+1)} .
    \end{align*}
    故 $X + Y\sim NB(r+1, p)$, 证毕.
    \item 注意到 分布$G(p)$ 即为 $NB(1, p)$. 我们用数学归纳法, 当 $r = 2$ 时由 (1) 中结论有 \[X_1 + X_2 \sim NB(2, p)\] 成立.  
    假设当 $r = k$ 其中 $k \in \N_{\ge 2}$ 时有 $\sum_{i=1}^{k}X_i \sim NB(k,p)$. 

    考虑 $r = k + 1$ 那么 由于$\sum_{i=1}^{k}X_i \sim NB(k,p)$ 及(1)中结论有 
    \[
        \sum_{i=1}^{k+1} X_i = \left(\sum_{i=1}^{k}X_i\right) + X_{k+1} \sim NB(k+1, p)
    \]
    故由数学归纳法, 对于正整数 $r$, 随机变量$X_1, \cdots, X_r \iid G(p)$,  $\sum_{i=1}^{r}X_i \sim NB(r,p)$. 证毕.
\end{enumerate}
\end{answer}
\end{document}
