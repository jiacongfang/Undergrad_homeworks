
\documentclass[11pt]{article}           
\usepackage[UTF8]{ctex}
\usepackage[a4paper]{geometry}
\geometry{left=2.0cm,right=2.0cm,top=2.5cm,bottom=2.25cm}

\usepackage{xcolor}
\usepackage{paralist}
\usepackage{enumitem}
\setenumerate[1]{itemsep=3pt,partopsep=0pt,parsep=0pt,topsep=0pt}
\setitemize[1]{itemsep=0pt,partopsep=0pt,parsep=0pt,topsep=0pt}
\usepackage{comment}
\usepackage{booktabs}
\usepackage{graphicx}
\usepackage{float}
\usepackage{sgame} % For Game Theory Matrices 
% \usepackage{diagbox} % Conflict with sgame
\usepackage{amsmath,amsfonts,graphicx,amssymb,bm,amsthm}
%\usepackage{algorithm,algorithmicx}
\usepackage[ruled]{algorithm2e}
\usepackage[noend]{algpseudocode}
\usepackage{fancyhdr}
\usepackage{tikz}
\usepackage{pgfplots}
\pgfplotsset{compat=1.18}
\usepackage{graphicx}
\usetikzlibrary{arrows,automata}
\usepackage[hidelinks]{hyperref}
\usepackage{extarrows}
\usepackage{totcount}
\setlength{\headheight}{14pt}
\setlength{\parindent}{0 in}
\setlength{\parskip}{0.5 em}
\usepackage{helvet}
\usepackage{dsfont}
% \usepackage{newtxmath}
\usepackage[labelfont=bf]{caption}
\renewcommand{\figurename}{Figure}
\usepackage[english]{babel}
\usepackage{datetime}
\usepackage{lastpage}
\usepackage{istgame}
\usepackage{sgame}
\usepackage{tcolorbox}
% \newdateformat{mydate}{\shortmonthname[\THEMONTH]. \THEDAY \THEYEAR}

\newtheorem{theorem}{Theorem}
\newtheorem{lemma}[theorem]{Lemma}
\newtheorem{proposition}[theorem]{Proposition}
\newtheorem{claim}[theorem]{Claim}
\newtheorem{corollary}[theorem]{Corollary}
\newtheorem{definition}[theorem]{Definition}
\newtheorem*{definition*}{Definition}
\newtheorem{remark}[theorem]{Remark}

\newenvironment{problem}[2][Problem]{\begin{trivlist}
    \item[\hskip \labelsep {\bfseries #1}\hskip \labelsep {\bfseries #2.}]\songti}{\hfill$\blacktriangleleft$\end{trivlist}}
\newenvironment{answer}[1][Solution]{\begin{trivlist}
\item[\hskip \labelsep {\bfseries #1.}\hskip \labelsep]}{\hfill$\lhd$\end{trivlist}}

\newcommand\1{\mathds{1}}
% \newcommand\1{\mathbf{1}}
\newcommand\R{\mathbb{R}}
\newcommand\E{\mathbb{E}}
\newcommand\N{\mathbb{N}}
\newcommand\NN{\mathcal{N}}
\newcommand\per{\mathrm{per}}
\newcommand\PP{\mathbb{P}}
\newcommand\dd{\mathrm{d}}
\newcommand\Var{\mathrm{Var}}
\newcommand\Cov{\mathrm{Cov}}
\newcommand{\Exp}{\mathrm{Exp}}
\newcommand{\arrp}{\xrightarrow{P}}
\newcommand{\arrd}{\xrightarrow{d}}
\newcommand{\arras}{\xrightarrow{a.s.}}
\newcommand{\arri}{\xrightarrow{n\rightarrow\infty}}

\definecolor{lightgray}{gray}{0.75}

\DeclareMathOperator*{\argmax}{argmax} % 定义 \argmax 运算符
\DeclareMathOperator*{\argmin}{argmin} % 定义 \argmin 运算符
\title{Homework \#3}
\usetikzlibrary{positioning}

\begin{document}
\kaishu

\pagestyle{fancy}
\lhead{\CJKfamily{zhkai} Peking University}
\chead{\kaishu }
\rhead{\CJKfamily{zhkai} Machine Learning, 2024 Fall}
\fancyfoot[R]{} 
\fancyfoot[C]{\thepage\ /\ \pageref{LastPage} \\ \textcolor{lightgray}{Last Compile: \today}}
% \regtotcounter{page}
% \fancyfoot[C]{\kaishu 第\thepage 页共\totvalue{page}页}


\begin{center}
    {\LARGE \bf Homework 4}

    {Name: 方嘉聪\ \  ID: 2200017849}            % Write down your name and ID here.
\end{center}

\begin{problem}{1}
    对于如下原问题和对偶问题, 其中 $f(x), g(x)$ 为凸函数, $h(x)$ 为线性函数. 
    \begin{align*}
        \begin{aligned}
            \mathbf{(P)}\quad \min_x \,\,  & f(x) \\
            \text{s.t.} \,\, & g_i(x) \le 0 \\
            & h_i(x) = 0
        \end{aligned}
        && \text{and} &&
        \begin{aligned}
            \mathbf{(D)}\quad \max_{\lambda, \mu} \,\,  & \mathcal{L}(\varphi(\lambda, \mu); \lambda, \mu) \\
            \text{s.t.} \,\, & \lambda \ge 0.
        \end{aligned}
    \end{align*}
    其中
    \begin{align*}
        \mathcal{L}(x;\lambda, \mu) := f(x) + \sum_i \lambda_i g_i(x) + \sum_i \mu_i h_i(x). \quad \varphi(\lambda, \mu) := \argmin_x \mathcal{L}(x;\lambda, \mu).
    \end{align*}
    以下为关于 $x^*, \lambda^*, \mu^*$ 的 KKT condition:
    \begin{alignat}{2}
        \label{eq:1}
        &\textbf{1.Stationary:} \quad  &\left.\frac{\partial \mathcal{L}}{\partial x}\right|_{x^*, \lambda^*, \mu^*}  =  0 \\
        \label{eq:2}
        &\textbf{2.Primal feasible:} \quad &g_i(x^*)  \leq  0,\quad h_i(x^*) = 0 \\
        \label{eq:3}
        &\textbf{3.Dual feasible:} \quad  &\lambda_i^*  \geq  0 \\
        \label{eq:4}
        &\textbf{4.Complementary slackness:} \quad  &\lambda_i^* g_i(x^*)  =  0.
    \end{alignat}
    请证明:
    \begin{enumerate}
        \item $x^*, \lambda^*, \mu^*$ 满足 KKT condition 是 $x^*, \lambda^*, \mu^*$ 为原问题和对偶问题的最优解的必要条件. 即若 $x^*, \lambda^*, \mu^*$ 为原问题和对偶问题的最优解, 则 $x^*, \lambda^*, \mu^*$ 满足 KKT condition.
        \item (Optional) KKT condition 是 $x^*, \lambda^*, \mu^*$ 为原问题和对偶问题的最优解的充分条件.
    \end{enumerate}
\end{problem}

\begin{answer}
\begin{enumerate}
    \item (必要条件) 若 $x^*, \lambda^*, \mu^*$ 为原问题和对偶问题的最优解, \eqref{eq:2} 和 \eqref{eq:3} 显然成立. 
    对于 \eqref{eq:1}, 结合课上已经完成的推导, 最优解满足 $x^* = \varphi(\lambda^*, \mu^*) = \argmin_x \mathcal{L}(x;\lambda^*, \mu^*)$, 即
    \begin{align*}
        \left.\frac{\partial \mathcal{L}}{\partial x}\right|_{x^*, \lambda^*, \mu^*} = 0.
    \end{align*}
    对于 \eqref{eq:4}, 记 Lagrange dual function 为 
    \begin{align*}
        s(\lambda, \mu) := \inf_x \mathcal{L}(x;\lambda, \mu) = \inf_x \left(f(x) + \sum_i \lambda_i g_i(x) + \sum_i \mu_i h_i(x)\right).
    \end{align*}
    由于 $f(x), g(x)$ 为 凸函数, $h(x)$ 为线性函数, 故满足强对偶性, 即$f(x^*) = s(\lambda^*, \mu^*)$. 进一步有
    \begin{align*}
        f(x^*) &= s(\lambda^*, \mu^*) = \inf_x \left(f(x) + \sum_i \lambda_i^* g_i(x) + \sum_i \mu_i^* h_i(x)\right) \\
        &\le f(x^*) + \sum_i \lambda_i^* g_i(x^*) + \sum_i \mu_i^* h_i(x^*) \\
        &\le f(x^*).
    \end{align*}
    故有
    \begin{align*}
        \sum_{i} \lambda_i^* g_i(x^*) + \sum_{i} \mu_i^* h_i(x^*) = 0. \implies \sum_i \lambda_i^* g_i(x^*) = 0.
    \end{align*}
    由于 $\forall i, \lambda_i^* g_i(x^*)\le 0$, 故 $\lambda_i^* g_i(x^*) = 0, \forall i$. 综上, $x^*, \lambda^*, \mu^*$ 满足 KKT condition.
    \item (充分条件) 由于 $x^*, \lambda^*, \mu^*$ 满足 \eqref{eq:2} 和 \eqref{eq:3}. 故 $x^*$ 和 $\lambda^*,\mu^*$ 分别为原问题和对偶问题的可行解. 
    
    设原问题的最优解为 $x_\text{opt}$, 记 $f(x_\text{opt}) = f^*$. 可以证明(即弱对偶性), 对于任意 $\lambda \ge 0$ 和 $\mu$, 有
    \begin{align}
        \label{eq:5}
        s(\lambda, \mu) \le f^*.
    \end{align}
    取 $\lambda = \lambda^*, \mu = \mu^*$, 有
    \begin{align*}
        s(\lambda^*, \mu^*) \le f^*.
    \end{align*}
    而由于 $x^*, \lambda^*, \mu^*$ 满足 \eqref{eq:1}, 那么
    \begin{align*}
        x^* = \varphi(\lambda^*, \mu^*) = \argmin_x \mathcal{L}(x;\lambda^*, \mu^*).
    \end{align*}
    故
    \begin{align*}
        s(\lambda^*, \mu^*) &= \inf_x \mathcal{L}(x;\lambda^*, \mu^*) \\
        &= f(x^*) + \sum_i \lambda_i^* g_i(x^*) + \sum_i \mu_i^* h_i(x^*) \\
        &= f(x^*) \le f^*.
    \end{align*}
    由于 $x^*$ 为原问题的可行解, 故 $f(x^*) \ge f^*$, 故 $f(x^*) = f^*$. 即 $x^*$ 为原问题的最优解. 结合 \eqref{eq:5}, 可知$\lambda^*, \mu^*$ 为对偶问题的最优解.
    
    综上, $x^*, \lambda^*, \mu^*$ 为原问题和对偶问题的最优解.
\end{enumerate}
\end{answer}

\end{document}