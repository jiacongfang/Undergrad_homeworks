\documentclass[11pt]{article}           
\usepackage[UTF8]{ctex}
\usepackage[a4paper]{geometry}
\geometry{left=2.0cm,right=2.0cm,top=2.5cm,bottom=2.25cm}

\usepackage{xcolor}
\usepackage{paralist}
\usepackage{enumitem}
\setenumerate[1]{itemsep=1pt,partopsep=0pt,parsep=0pt,topsep=0pt}
\setitemize[1]{itemsep=0pt,partopsep=0pt,parsep=0pt,topsep=0pt}
\usepackage{comment}
\usepackage{booktabs}
\usepackage{graphicx}
\usepackage{float}
\usepackage{sgame} % For Game Theory Matrices 
% \usepackage{diagbox} % Conflict with sgame
\usepackage{amsmath,amsfonts,graphicx,amssymb,bm,amsthm}
%\usepackage{algorithm,algorithmicx}
\usepackage{algorithm,algorithmicx}
\usepackage[noend]{algpseudocode}
\usepackage{fancyhdr}
\usepackage{tikz}
\usepackage{pgfplots}
\pgfplotsset{compat=1.18}
\usepackage{graphicx}
\usetikzlibrary{arrows,automata}
\usepackage[hidelinks]{hyperref}
\usepackage{extarrows}
\usepackage{totcount}
\setlength{\headheight}{14pt}
\setlength{\parindent}{0 in}
\setlength{\parskip}{0.5 em}
\usepackage{helvet}
\usepackage{dsfont}
% \usepackage{newtxmath}
\usepackage[labelfont=bf]{caption}
\renewcommand{\figurename}{Figure}
\usepackage{lastpage}
\usepackage{istgame}
\usepackage{tcolorbox}
% \newdateformat{mydate}{\shortmonthname[\THEMONTH]. \THEDAY \THEYEAR}

\RequirePackage{algorithm}

\makeatletter
\newenvironment{algo}
  {% \begin{breakablealgorithm}
    \begin{center}
      \refstepcounter{algorithm}% New algorithm
      \hrule height.8pt depth0pt \kern2pt% \@fs@pre for \@fs@ruled
      \parskip 0pt
      \renewcommand{\caption}[2][\relax]{% Make a new \caption
        {\raggedright\textbf{\fname@algorithm~\thealgorithm} ##2\par}%
        \ifx\relax##1\relax % #1 is \relax
          \addcontentsline{loa}{algorithm}{\protect\numberline{\thealgorithm}##2}%
        \else % #1 is not \relax
          \addcontentsline{loa}{algorithm}{\protect\numberline{\thealgorithm}##1}%
        \fi
        \kern2pt\hrule\kern2pt
     }
  }
  {% \end{breakablealgorithm}
     \kern2pt\hrule\relax% \@fs@post for \@fs@ruled
   \end{center}
  }
\makeatother


\newtheorem{theorem}{Theorem}
\newtheorem{lemma}[theorem]{Lemma}
\newtheorem{proposition}[theorem]{Proposition}
\newtheorem{claim}[theorem]{Claim}
\newtheorem{corollary}[theorem]{Corollary}
\newtheorem{definition}[theorem]{Definition}
\newtheorem*{definition*}{Definition}

\newenvironment{problem}[2][Problem]{\begin{trivlist}
    \item[\hskip \labelsep {\bfseries #1}\hskip \labelsep {\bfseries #2.}]\songti}{\hfill$\blacktriangleleft$\end{trivlist}}
\newenvironment{answer}[1][Solution]{\begin{trivlist}
    \item[\hskip \labelsep {\bfseries #1.}\hskip \labelsep]}{\hfill$\lhd$\end{trivlist}}

\newcommand\1{\mathds{1}}
% \newcommand\1{\mathbf{1}}
\newcommand\R{\mathbb{R}}
\newcommand\E{\mathbb{E}}
\newcommand\N{\mathbb{N}}
\newcommand\NN{\mathcal{N}}
\newcommand\per{\mathrm{per}}
\newcommand\PP{\mathbb{P}}
\newcommand\dd{\mathrm{d}}
\newcommand\ReLU{\mathrm{ReLU}}
\newcommand{\Exp}{\mathrm{Exp}}
\newcommand{\arrp}{\xrightarrow{P}}
\newcommand{\arrd}{\xrightarrow{d}}
\newcommand{\arras}{\xrightarrow{a.s.}}
\newcommand{\arri}{\xrightarrow{n\rightarrow\infty}}
\newcommand{\iid}{\overset{\text{i.i.d}}{\sim}}

% New math operators
\DeclareMathOperator{\sgn}{sgn}
\DeclareMathOperator{\diag}{diag}
\DeclareMathOperator{\rank}{rank}
\DeclareMathOperator{\tr}{tr}
\DeclareMathOperator{\Var}{Var}
\DeclareMathOperator{\Cov}{Cov}
\DeclareMathOperator{\Corr}{Corr}
\DeclareMathOperator{\MSE}{MSE}
\DeclareMathOperator{\Bias}{Bias}
\DeclareMathOperator*{\argmax}{argmax}
\DeclareMathOperator*{\argmin}{argmin}


\definecolor{lightgray}{gray}{0.75}


\begin{document}
\kaishu

\pagestyle{fancy}
\lhead{\CJKfamily{zhkai} 北京大学}
\chead{}
\rhead{\CJKfamily{zhkai} 2024年秋\ 信息学中的概率统计(王若松)}
\fancyfoot[R]{} 
\fancyfoot[C]{\thepage\ /\ \pageref{LastPage} \\ \textcolor{lightgray}{最后编译时间: \today}}

\begin{center}
    {\LARGE \bf Final Review}
\end{center}

\begin{problem}{1. 多元随机变量}
    $X_1, \cdots, X_n \sim \Exp(\theta_1), \cdots ,\Exp(\theta_n)$. 求
    \begin{align*}
        Y = \max\left\{\frac{a_1}{X_1}, \cdots, \frac{a_n}{X_n}\right\}
    \end{align*}
    的概率密度函数.
    \textcolor{red}{对于多元连续随机变量关注一下 min/max 的计算等等.}
\end{problem}

\begin{answer}
    \begin{align*}
        \PP(Y \le y) = \prod_{i=1}^{n} \PP\left(\frac{a_i}{X_i} \le y\right) = \prod_{i=1}^{n} \PP\left(X_i \ge \frac{a_i}{y}\right)
    \end{align*}
\end{answer}

\begin{problem}{2. Tail inequality}
    关注 Chernoff-Hoeffding 的 Chernoff Bound 形式.
\end{problem}

\begin{problem}{3. Law of Large Number}
    关注一下大数定律的各个条件
    \begin{itemize}
        \item Chebyshev大数定律.
        \item 辛钦大数定律. $\{X_n\}$ 独立同分布
    \end{itemize}
    举例的一个问题是重要性采样.

    Chernoff Bound 可能要出一道压轴题, 比较看数学直觉.
\end{problem}

\begin{problem}{4. 参数估计}
    点估计和极大似然估计, 要会计算, 会出一些特殊分布的参数估计计算. 下面是一些例子:
    \begin{gather*}
        X = e^Y, Y \sim \NN(\mu, \sigma^2) \implies \hat{\mu} = \frac{1}{n} \sum_{i=1}^n \log X_i. \\
        \text{(反Gamma分布)}, f(x) = \theta x^{-2} \exp(-\theta/x). \\
        \text{(指数分布族)}, f(x) = \exp(\theta h(x) - H(x)) \cdot g(x) \implies \hat{\theta}_n = h^{-1}\left(\frac{1}{n}\sum_{i=1}^n h(x_i)\right). 
    \end{gather*}
    矩估计. 无偏估计量, 渐进无偏估计量, 一致估计量, 均方误差. \textcolor{red}{计算的速度}.
\end{problem}

\begin{problem}{5. 假设检验}
    对于显著性水平$\alpha$, 建立假设, 找到一个统计量, 计算拒绝域$c(\alpha)$. 例如分成两问, 一问先算算统计量, 后面去计算拒绝域. 

    从广义似然比的视角来理解, 计算似然函数 $L(\theta)$, 设 $H_0: \theta \in \Theta_1, H_0 \in \Theta_1$, 那么拒绝域也可以写成:
    \begin{align*}
        W = \left\{\frac{\sup_{\theta\in\Theta_0} L(\theta; x_1, \cdots, x_n)}{\sup_{\theta\in\Theta_1} L(\theta; x_1, \cdots, x_n)} \le \lambda\right\}.
    \end{align*}
    例如, 考虑一个均匀分布 $U(0, \theta)$, 样本$x_1, \cdots, x_n$, 那么 $\theta$ 的极大似然估计为 $\hat{\theta} = \max\{x_1, \cdots, x_n\}$. 之后再进行假设检验, 求拒绝域. 可以再加上与矩估计量有效性的比较(无偏, 一致, MSE).
\end{problem}

\begin{problem}{6. 区间估计}
    大概的一个形式是, 给定显著性水平 $\alpha$, 估计 $\theta$ 的区间 $[L, U]$, 使得
    \begin{align*}
        \PP(\theta \in [L, U]) \ge 1 - \alpha.
    \end{align*}
    需要找一下枢轴量, 由待估计量组成的函数, 且其分布不依赖于待估计量. 或是用 Chernoff Bound 来估计.
\end{problem}

\begin{problem}{7. 回归分析}
    最小二乘估计
    \begin{align*}
        Q(\alpha, \beta) = \sum_{i=1}^n (y_i - \alpha - \beta x_i)^2.
    \end{align*}
    $Q(\alpha, \beta)$ 可能会加上一些正则项. 解正规方程 
    \begin{align*}
        \frac{\partial Q}{\partial \alpha} = 0, \quad \frac{\partial Q}{\partial \beta} = 0.
    \end{align*}

    除了最小二乘估计, 还可以使用极大似然估计. 例如, 存在一个未知的常数 $\eta_i$, 且有
    \begin{align*}
        y_i &= \alpha + \beta \eta_i + \varepsilon_i \sim \NN(\alpha + \beta \eta_i, \sigma_{\varepsilon}^2). \\
        x_i &= \eta_i + \xi_i \sim \NN(\eta_i, \sigma_{\xi}^2).
    \end{align*}
    那么极大似然估计为
    \begin{align*}
        L(\alpha, \beta, \sigma_\varepsilon, \sigma_\xi) = \frac{1}{(2\pi)^n}\frac{\lambda^{n/2}}{\sigma_\xi^{2n}} \exp(\cdot)
    \end{align*}
    先微分求一下 $\eta_i$ 的极大似然估计, 带入之后, 再求 $\alpha, \beta, \sigma_\varepsilon, \sigma_\xi$ 的极大似然估计.

    之后会再求一些期望, MSE, Covariance 等等. 这里可以考虑将
    \begin{align*}
        \hat{\beta} = c_1 \beta + c_2 \varepsilon + c_3 \xi 
    \end{align*}
    之后再计算可能会简单一些.
\end{problem}


\end{document}