
\documentclass[11pt]{article}           
\usepackage[UTF8]{ctex}
\usepackage[a4paper]{geometry}
\geometry{left=2.0cm,right=2.0cm,top=2.5cm,bottom=2.25cm}

\usepackage{xcolor}
\usepackage{paralist}
\usepackage{enumitem}
\setenumerate[1]{itemsep=0pt,partopsep=0pt,parsep=0pt,topsep=0pt}
\setitemize[1]{itemsep=0pt,partopsep=0pt,parsep=0pt,topsep=0pt}
\usepackage{comment}
\usepackage{booktabs}
\usepackage{graphicx}
\usepackage{float}
\usepackage{sgame} % For Game Theory Matrices 
% \usepackage{diagbox} % Conflict with sgame
\usepackage{amsmath,amsfonts,graphicx,amssymb,bm,amsthm}
%\usepackage{algorithm,algorithmicx}
\usepackage[ruled]{algorithm2e}
\usepackage[noend]{algpseudocode}
\usepackage{fancyhdr}
\usepackage{tikz}
\usepackage{pgfplots}
\pgfplotsset{compat=1.18}
\usepackage{graphicx}
\usetikzlibrary{arrows,automata}
\usepackage[hidelinks]{hyperref}
\usepackage{extarrows}
\usepackage{totcount}
\setlength{\headheight}{14pt}
\setlength{\parindent}{0 in}
\setlength{\parskip}{0.5 em}
\usepackage{helvet}
\usepackage{dsfont}
% \usepackage{newtxmath}
\usepackage[labelfont=bf]{caption}
\renewcommand{\figurename}{Figure}
\usepackage[english]{babel}
\usepackage{datetime}
\usepackage{lastpage}
\usepackage{istgame}
\usepackage{sgame}
\usepackage{tcolorbox}
% \newdateformat{mydate}{\shortmonthname[\THEMONTH]. \THEDAY \THEYEAR}

\newtheorem{theorem}{Theorem}
\newtheorem{lemma}[theorem]{Lemma}
\newtheorem{proposition}[theorem]{Proposition}
\newtheorem{claim}[theorem]{Claim}
\newtheorem{corollary}[theorem]{Corollary}
\newtheorem{definition}[theorem]{Definition}
\newtheorem*{definition*}{Definition}

\newenvironment{problem}[2][Problem]{\begin{trivlist}
    \item[\hskip \labelsep {\bfseries #1}\hskip \labelsep {\bfseries #2.}]\songti}{\hfill$\blacktriangleleft$\end{trivlist}}
\newenvironment{answer}[1][Solution]{\begin{trivlist}
\item[\hskip \labelsep {\bfseries #1.}\hskip \labelsep]}{\hfill$\lhd$\end{trivlist}}

\newcommand\1{\mathds{1}}
% \newcommand\1{\mathbf{1}}
\newcommand\R{\mathbb{R}}
\newcommand\E{\mathbb{E}}
\newcommand\N{\mathbb{N}}
\newcommand\NN{\mathcal{N}}
\newcommand\per{\mathrm{per}}
\newcommand\PP{\mathbb{P}}
\newcommand\dd{\mathrm{d}}
\newcommand\Var{\mathrm{Var}}
\newcommand\Cov{\mathrm{Cov}}
\newcommand{\Exp}{\mathrm{Exp}}
\newcommand{\arrp}{\xrightarrow{P}}
\newcommand{\arrd}{\xrightarrow{d}}
\newcommand{\arras}{\xrightarrow{a.s.}}
\newcommand{\arri}{\xrightarrow{n\rightarrow\infty}}

\definecolor{lightgray}{gray}{0.75}

\DeclareMathOperator*{\argmax}{argmax} % 定义 \argmax 运算符
\title{Homework \#3}
\usetikzlibrary{positioning}

\begin{document}
\kaishu

\pagestyle{fancy}
\lhead{\CJKfamily{zhkai} Peking University}
\chead{\kaishu }
\rhead{\CJKfamily{zhkai} Machine Learning, 2024 Fall}
\fancyfoot[R]{} 
\fancyfoot[C]{\thepage\ /\ \pageref{LastPage} \\ \textcolor{lightgray}{Last Compile: \today}}
% \regtotcounter{page}
% \fancyfoot[C]{\kaishu 第\thepage 页共\totvalue{page}页}


\begin{center}
    {\LARGE \bf Homework 2}

    {Name: 方嘉聪\ \  ID: 2200017849}            % Write down your name and ID here.
\end{center}

\begin{problem}{1}
    已知 $(x_1, y_1), (x_2, y_2), \cdots, (x_n, y_n)$, 其中 $x_i \in \R^m$, $y_i \in \{\pm 1\}$.
    证明以下两个优化问题的等价性。
    \begin{align*}
        \begin{aligned}
            \max \,\,  & t \\
            \text{s.t.} \,\, & y_i(w^\top x_i + b) \geq t, \forall i\in[n],  \\
            & ||w||_2 = 1.
        \end{aligned}
        && \text{与} &&
        \begin{aligned}
            \min \,\,  & ||w||_2^2 \\
            \text{s.t.} \,\, & y_i(w^\top x_i + b) \geq 1, \forall i\in[n].
        \end{aligned}
    \end{align*}

    即说明两个问题的解的关系,以及两个问题求出的最优解$(w, b)$ 之间的关系。
\end{problem}
\begin{answer}
我们从左侧问题出发逐步转化到右侧问题。首先,左侧问题的约束条件可以等价地写为
\begin{align*}
    y_i(w^\top x_i + b) \geq t \land ||w||_2 = 1 \iff y_i\left(\frac{w^\top}{||w||_2} \cdot x_i + b\right) \ge t \iff y_i\left(w^\top x_i + b'\right) \ge t\cdot ||w||_2.
\end{align*}
其中 $w^\top \in \R^n, b, b' \in \R$. 注意到上述不等式两边齐次, 那么我们可以令 $t\cdot||w||_2 := 1$. 那么有 
\begin{align*}
    \max t \iff \max \frac{1}{||w||_2} \iff  \min ||w||_2 \iff \min ||w||_2^2.
\end{align*}
类似的, 从右侧的优化问题出发转化到左侧问题. 因此, 左侧问题的求解等价于求解右侧问题. 设左侧问题的最优解为 $(w_1, b_1)$, 右侧问题的最优解为 $(w_2, b_2)$, 由上述推导过程有: 
\begin{align*}
    w_1 = \frac{w_2}{||w_2||_2}, \quad b_1 = \frac{b_2}{||w_2||_2}, \quad t = \frac{1}{||w_2||_2}.
\end{align*}
证毕.
\end{answer}

\end{document}