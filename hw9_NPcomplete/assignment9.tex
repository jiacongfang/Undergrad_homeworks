\documentclass[11pt]{article}
\usepackage[UTF8]{ctex}
\usepackage[a4paper]{geometry}
\geometry{left=2.0cm,right=2.0cm,top=2.5cm,bottom=2.5cm}

\usepackage{caption}
\usepackage{paralist}
\usepackage{enumitem}
\setenumerate[1]{itemsep=0pt,partopsep=0pt,parsep=0pt,topsep=0pt}
\setitemize[1]{itemsep=0pt,partopsep=0pt,parsep=0pt,topsep=0pt}
\usepackage{comment}
\usepackage{booktabs}
\usepackage{graphicx}
\usepackage{float}
\usepackage{diagbox}
\usepackage{amsmath,amsfonts,graphicx,amssymb,bm,amsthm}
\usepackage{algorithm,algorithmicx}
% \usepackage[ruled, linesnumbered]{algorithm2e}
% \usepackage[linesnumbered]{algorithm2e}
\usepackage[noend]{algpseudocode}
\usepackage{fancyhdr}
\usepackage{tikz}
\usepackage{graphicx}
\usetikzlibrary{arrows,automata,positioning}
\usepackage{hyperref}
\usepackage{extarrows}
\usepackage{wrapfig}
% 这是一些字体选项
\usepackage{helvet}
% \usepackage{mathpazo}
\usepackage{fontspec}
% \setmainfont{Times New Roman}
% \setmainfont{Comic Sans MS} % 比较fancy的字体
% \setmainfont{Avenir}
% \setmainfont{Palatino}

\setlength{\headheight}{14pt}
\setlength{\parindent}{0 in}
\setlength{\parskip}{0.5 em}


\newtheorem{theorem}{Theorem}
\newtheorem{lemma}[theorem]{Lemma}
\newtheorem{proposition}[theorem]{Proposition}
\newtheorem{claim}[theorem]{Claim}
\newtheorem{corollary}[theorem]{Corollary}
\newtheorem{definition}[theorem]{Definition}
\newtheorem*{definition*}{Definition}

% \newenvironment{problem}[2][Problem]{\begin{trivlist}
% \item[\hskip \labelsep{\bfseries#1}\hskip\labelsep{\bfseries#2.}]}{\hfill$\blacktriangleleft$\end{trivlist}}
% 标题后强制换行
\newenvironment{problem}[2][Problem]{\begin{trivlist}
    \item[\hskip \labelsep{\bfseries#1}\hskip\labelsep{\bfseries#2.}]\mbox{}\newline}{\hfill$\blacktriangleleft$\end{trivlist}}
\newenvironment{answer}[1][Answer]{\begin{trivlist}
\item[\hskip \labelsep{\bfseries\itshape#1.}\hskip \labelsep]}{\hfill$\lhd$\end{trivlist}}

\DeclareMathOperator*{\minimize}{minimize}
\DeclareMathOperator*{\maximize}{maximize}
\newcommand\E{\mathbb{E}}
\newcommand\per{\mathrm{per}}
\renewcommand{\algorithmicrequire}{\textbf{Input:}}
\renewcommand{\algorithmicensure}{\textbf{Output:}}
\algrenewcommand{\algorithmiccomment}[1]{\hfill $//$ #1}
% chktex-file 44
% \renewcommand{\familydefault}{\sfdefault}

\RequirePackage{algorithm}

\makeatletter
\newenvironment{algo}
  {% \begin{breakablealgorithm}
    \begin{center}
      \refstepcounter{algorithm}% New algorithm
      \hrule height.8pt depth0pt \kern2pt% \@fs@pre for \@fs@ruled
      \parskip 0pt
      \renewcommand{\caption}[2][\relax]{% Make a new \caption
        {\raggedright\textbf{\fname@algorithm~\thealgorithm} ##2\par}%
        \ifx\relax##1\relax % #1 is \relax
          \addcontentsline{loa}{algorithm}{\protect\numberline{\thealgorithm}##2}%
        \else % #1 is not \relax
          \addcontentsline{loa}{algorithm}{\protect\numberline{\thealgorithm}##1}%
        \fi
        \kern2pt\hrule\kern2pt
     }
  }
  {% \end{breakablealgorithm}
     \kern2pt\hrule\relax% \@fs@post for \@fs@ruled
   \end{center}
  }
\makeatother

% set for automata
\tikzset{>=stealth',shorten >=1pt,auto,node distance=2cm, % Increase node distance to 4cm
                    thick,main node/.style={circle,draw,font=\sffamily\Large\bfseries}}


\title{Homework \#9}
\usetikzlibrary{positioning}

\begin{document}
\captionsetup[figure]{labelfont={bf},name={Fig.},labelsep=period}
\kaishu

\pagestyle{fancy}
\lhead{\CJKfamily{zhkai} Peking University}
\chead{}
\rhead{\CJKfamily{zhkai} Algorithm Design and Analysis (Honor Track)}

\begin{center}
    {\LARGE \bf Homework 9}\\
    {Name: 方嘉聪\ \  ID: 2200017849}            % Write down your name and ID here.
\end{center}

\begin{problem}{1.(Search and Decision Problems)}
    As discussed in class, NP is a class of decision problems, i.e., the answer is either ``yes'' or ``no''. This choice is convenient and often also captures the difficulty of searching for a solution. In our class, we have shown the reduction from search to decision for some problems. Here, you will show the search-to-decision reduction for two more examples.

    (1) \textbf{3-SAT}: A 3-SAT formula consists of $m$ clauses, each of which is a disjunction of three literals (where a literal is one of $n$ variables or its negation). We say a 3-SAT formula $F$ is satisfiable if there is an assignment of the $n$ variables such that $F$ evaluates to true. The 3-SAT problem asks us to decide whether a given $F$ is satisfiable.
    
    Suppose you are given a black box for a function $\mathsf{3SAT}$ that determines whether a given 3-SAT formula is satisfiable with timing cost $T_{\mathsf{3SAT}}$. Design an algorithm that finds a satisfying assignment for $F$ (assuming it is satisfiable) using $O(n)$ calls to $\mathsf{3SAT}$ and polynomially many other steps. Prove that your algorithm is correct and analyze its running time.
    
    (2) \textbf{3-Coloring}: We say an undirected graph is 3-colorable if there is an assignment of the colors $\{r, g, b\}$ to the vertices (a coloring) such that no two adjacent vertices have the same color. The 3-Coloring problem asks us to decide whether a given graph is 3-colorable.
    
    Suppose you are given a black box for a function $\mathsf{3Color}$ that determines whether a given graph is 3-colorable with timing cost $T_{\mathsf{3Color}}$. Design an algorithm that finds a coloring of a given graph using $O(n)$ calls to $\mathsf{3Color}$ and polynomially many other steps, where $n$ is the number of vertices in the input graph. Prove that your algorithm is correct and analyze its running time.
\end{problem}
\begin{answer}
    (1) 只需要对$n$个变量依次赋值, 考虑如下的算法:
    \begin{algo}
        \caption{ \textbf{Find a satisfying assignment for 3-SAT}}
        \begin{algorithmic}[1] % The number tells where the line numbering should start
            \Require a 3-SAT formula $F$ with $n$ variables, $m$ clauses.
            \Ensure a satisfying assignment $S$ for $F$.
            \State 把 $F$ 中的一个变量$x_1$赋值为$1$. 
            \State \quad 将$F$中包含$x_1$的子句删去, 对于包含$\neg x_1$的子句$(\neg x_1 \lor y \lor z)$, 替换为$(y\lor y\lor z)$其中$(y\neq \neg x_1)$.
            \State 设新得到formula记为$F'$. 调用$\texttt{3SAT}(F')$, 记结果为$R$.
            \If {$R = 1$}
            \State $\texttt{S.append}(x_1=1)$, 对$F'$递归调用第1行(若为空直接返回$S$). 
            \Else 
            \State $\texttt{S.append}(x_1=0)$
            \State 将$F$中包含$\neg x_1$的子句删去, 对于包含$x_1$的子句$(x_1 \lor y \lor z)$, 替换为$(y\lor y\lor z)$其中$(y\neq x_1)$.
            \State 记这样得到的formula为$F''$, 对$F''$递归调用第1行(若为空直接返回$S$).
            \EndIf
            \State \Return $S$
        \end{algorithmic}
    \end{algo}
    算法正确性: 第2行的算法时由于若$x_1 = 1$则包含$x_1$的子句赋值为1, 包含$\neg x_1$的子句赋值取决于子句内的其他变量. 第8行同理. 每一次调用$F$的变量数减1, 递归最多需要$n$次.

    时间复杂度: 总共调用$O(n)$次3SAT, 每一步只需遍历一遍formula用时$O(m)$, 是多项式时间的. 故总的时间复杂度为$O(nT_{3SAT} + mn)$

    (2) 设图$G = (V,E)$, $|V| = n$. 首先调用$\texttt{3Color}(G)$, 若返回否, 说明$G$不是3-colorable的. 若返回是, 向$G$中添加点$x_r,x_g,x_b$和边$(x_r,x_g),(x_r,x_b),(x_g,x_b)$, 记得到的图为$G_0$, 那么$G_0$也是3-colorable的. 下面我们通过递归的方式找到$G$的一个3-coloring. 按照$v_1, v_2, \cdots, v_n$的顺序操作每个点, 得到一个满足3-colorable的图序列: $G_0, G_1,G_2, \cdots, G_n$, 其中$G_i \rightarrow G_{i+1}$通过如下的步骤得到:
    \begin{itemize}
        \item 构造图$G_i^1 = G_i\cup\{(r,v_i), (b,v_i)\}$, $G_i^2 = G_i\cup\{(r,v_i), (g,v_i)\}$, $G_i^3 = G_i\cup\{(g,v_i), (b,v_i)\}$.
        \item 依次调用$\texttt{3Color}(G_i^j), (j=1,2,3)$.
        \item 若$\texttt{3Color}(G_i^1) = 1$, 那么$v_i$颜色可以为$g$, 令$G_{i+1} = G_i^1$. 
        \item 若$\texttt{3Color}(G_i^2) = 1$, 那么$v_i$颜色可以为$b$, 令$G_{i+1} = G_i^2$.
        \item 若上述均不为1, 那么$v_i$颜色可以为$r$, 令$G_{i+1} = G_i^3$.
    \end{itemize}
    最终得到了$G$得一个合法的3-coloring. 总共需要调用$O(2n+1) = O(n)$次\texttt{3Color}, 符合题意.
\end{answer} 

\begin{problem}{2. (3-Dimensional Matching)}
In our class, we introduced how to solve the bipartite matching problem in polynomial time. In particular, the perfect bipartite matching problem can be written in the following way: Suppose we are given two disjoint sets $A$ and $B$, each of size $n$, and a set $P$ of pairs drawn from $A \times B$. The goal is to determine whether there exists a set of $n$ pairs in $P$ such that each element in $A \cup B$ is contained in exactly one of these pairs.

As a generalization, consider the following 3-dimensional matching problem: Suppose we are given three disjoint sets $A, B$, and $C$, each of size $n$, and a set $T$ of triples drawn from $A \times B \times C$. The goal is to determine whether there exists a set of $n$ triples in $T$ such that each element in $A \cup B \cup C$ is contained in exactly one of these triples.

(1) Prove that 3-dimensional-matching $\leq_{P}$ Set-Cover. This implies that 3-dimensional-matching $\in$ NP.

(2) Prove that 3-SAT $\leq_{P}$ 3-dimensional-matching. Together with (1), this implies that the 3-dimensional-matching problem is NP-complete.
\end{problem}
\begin{answer}
    (1) 如果$\bigcup T \neq A\cup B\cup C$, 那么显然不存在3维匹配. 考虑如下的多项式规约$f$:
    \begin{align*}
        f: \left\langle S \right\rangle \rightarrow \left\langle \mathcal{U}, T, n \right\rangle, \text{ 其中 } S = \{S_1, \cdots, S_n \} \subseteq T, ~\mathcal{U} = A\cup B\cup C
    \end{align*}
    注意到若$S \in \text{3-dimensional-matching}$,那么$S$对应的集合族$\mathcal{F} = \{S_1, \cdots, S_n\}$恰好是$\mathcal{U}$的一个大小为$n$的Set-Cover. 若$S \notin \text{3-dimensional-matching}$, 显然$\mathcal{U}$不存在一个大小至少为$n$的Set-Cover. 那么
    \begin{align*}
        f(S) \in \text{3-dimensional-matching} \iff f(S) \in \text{Set-Cover}
    \end{align*}
    故3-dimensional-matching $\leq_{P}$ Set-Cover, 证毕. 进而3-dimensional-matching $\in$ NP.

    注: 直接从NP的定义出发, 考虑证书为$T$的一个$n$元子集$S$, 直接验证$S$是否是3-dimensional-matching即可. 明显验证时间是多项式的, 且$|S|$是多项式的. 进而有3-dimensional-matching $\in$ NP.

    (2) 考虑如下的规约$g$, 对于任意一个3CNF公式$\varphi$($n$个变量, $m$个子句), 按照如下步骤构造$g(\varphi)$: 
    \begin{itemize}
        \item 对于任意一个变量$x_i$, 添加点集$A_i, B_i$和三元组$t_{ij}$(令$k = m$):
        \begin{align*}
            A_i = \{a_{i,1}, a_{i,2}, \cdots, a_{i,2k}\},~ B_i = \{b_{i,1}, b_{i,2}, \cdots, b_{i,2k}\},~ t_{ij} = (a_{i,j}, a_{i, j+1}, b_{i,j})
        \end{align*}
        总共有$4nk$个点, $2nk$个三元组. 
        \item 对于任意赋值, 若$x_i = 1$, 那么将$t_{ij}$($j$为奇数)添加到匹配中. 若$\neg x_i = 1$, 则将$t_{ij}$($j$为偶数)添加到匹配中. 此时$\{A_i\}$都被添加进了匹配中, $\{B_i\}$还剩下$nk$个点未被添加进匹配中.
        \item 对任意一个子句$C_j = (x_{j_1}\lor x_{j_2}\lor x_{j_3})$, 添加点集$N_j= \{c_{j}, \hat{c}_j\}$与三元组$r_{j_1}, r_{j_2}, r_{j_3}$:
        \begin{align*}
            \forall i = \{1,2,3\},~ &r_{j_i} = (c_j, \hat{c}_j, b_{j_i,2j}), \text{ 若$x_{j_i}$为$x_{j_i}$的形式}. \\
            &r_{j_i} = (c_j, \hat{c}_j, b_{j_i,2j-1}), \text{ 若$x_{j_i}$为$\neg x_{j_i}$的形式}.
        \end{align*}
        总共添加了$2m$个点, $3m$个三元组.
        \item 对子句$C_j$的任意赋值, 若$x_{j_i} = 1$则将$r_{j_i}$添加到匹配中(若有多个为1,则任选一个添加). $\{C_j\}$都被添加进了匹配中, 另外添加了$m$个$\{B_i\}$中的点.
        \item 注意到如上操作后, $\{B_i\}$还有$nk - m$个点未被添加进匹配中. 为了变成3-dimensional-matching, 我们额外添加$nk-m$个二元组$(e_i, \hat{e}_i)$, 每个这样的二元组与$\{B_i\}$中未添加进匹配中的点一一对应. 这样就构成了一个3-dimensional-matching, 其中:
        \begin{align*}
            A = \{a_{i,j}\mid j \text{ 为偶数}\}\cup\{c_j\} \cup\{e_j\},~ B = \{a_{i,j}\mid j \text{ 为奇数}\}\cup\{\hat{c}_j\} \cup\{\hat{e}_j\},~ C = \{b_{i,j}\}
        \end{align*}
    \end{itemize}
    规约显然是多项式时间的,下面简要证明规约的正确性:
    
    1.) 若$\varphi$是可满足的, 那么存在一个赋值使得$\varphi$中的每一个子句都为真. 对于每个子句,  这里$|A_i| = |B_i| = 2k$保证了不会有$b_{ij}$被重复添加进匹配中. 最后的添加的$nk-m$个二元组保证了$B$中的点都被添加进了匹配中. 故$g(\varphi)$是3-dimensional-matching的.

    2.) 若$A,B,C$是3-dimensional-matching的, 那么$x_i = 0/1$取决于$(c_j, \hat{c}_j, b_{i,j})$中$j$的奇偶性(偶数为1, 奇数为0), 那么由构造可以得到一组赋值使得$\varphi$为真. 故$g(\varphi)$是可满足的.

    综上所述, 3-SAT $\leq_{P}$ 3-dimensional-matching, 证毕. 
\end{answer}

\begin{problem}{3. (Restricted Monotone Satisfiability)}
    Consider an instance of the Satisfiability problem, specified by clauses $C_1, \ldots, C_k$ over a set of Boolean variables $x_1, \ldots, x_n$. \textbf{We say that the instance is \textit{monotone} if each literal is a non-negated variable (i.e., $x_i$ can appear as a literal but $\neg x_i$ cannot)}. Monotone instances are always satisfiable since we can simply set each variable to $1$.

For example, suppose we have the three clauses
$$
\left(x_{1} \vee x_{2}\right),\left(x_{1} \vee x_{3}\right),\left(x_{2} \vee x_{3}\right)
$$
This is monotone, and indeed the assignment that sets all three variables to 1 satisfies all the clauses. But we can observe that this is not the only satisfying assignment; we could also have set $x_{1}$ and $x_{2}$ to $1$, and $x_{3}$ to $0$. Indeed, for any monotone instance, it is natural to ask how few variables we need to set to 1 in order to satisfy~it.

Given a monotone instance of Satisfiability, together with a number $k$, the problem of \textit{Restricted Monotone Satisfiability} asks: is there a satisfying assignment for the instance in which at most $k$ variables are set to 1? Prove this problem is NP-complete.
\end{problem}

\begin{answer}
不妨简记\textit{Restricted Monotone Satisfiability}为RMS. 

(1) 首先证明RMS $\in$ NP. 输入为一个boolean formula $\varphi$和$k$, 考虑证书为一组下标(不超过$k$个), 长度显然是Poly($|\left\langle\varphi,k\right\rangle|$)的, 验证机只需先检验$\varphi$是否为monotone instance以及$k \le n$, 在检验令证书中的下标对应的变量赋1, $\varphi$是否可满足即可. 

(2) 再证明Set-Cover $\le_P$ RMS. 给定全集$\mathcal{U} = \{c_1, c_2, \cdots, c_t\}$, 子集族$\mathcal{S} = \{X_1, X_2, \cdots, X_n\}$, 判断是否存在大小不超过$k$的子集族$\mathcal{S}' \subseteq \mathcal{S}$使得$\bigcup \mathcal{S}' = \mathcal{U}$. 考虑如下规约$f:\left\langle \mathcal{U},\mathcal{S},k\right\rangle \rightarrow \left\langle \varphi,k \right\rangle$. 
\begin{itemize}
    \item 每个$X_i \in \mathcal{S}$对应着$\varphi$中一个变量$x_i$. 每个$c_i \in \mathcal{U}$对应着一个子句$C_i$. 
    \item 若$X_i = \{c_{i,1}, \cdots, c_{i,s}\}$, 则在子句$\{C_{i,1}, \cdots, C_{i,s}\}$中添加变量$x_i$.
    \item 设$\mathcal{S}' = \{X_{i_1}, \cdots, X_{i_{k'}}\}$, 其中$k' \le k$. 则令对应的变量$x_{i_1}, \cdots, x_{i_{k'}}$赋值为1, 其他为0.
\end{itemize}
$f$显然是多项式时间的. 下面证明规约的正确性:

1.) 若$\left\langle \mathcal{U},\mathcal{S},k\right\rangle \in$ Set-Cover. 那么由于$\mathcal{S}'$不交且$\bigcup \mathcal{S}' = \mathcal{U}$, 说明赋值$\{x_{i_1}, \cdots, x_{i_{k'}}\} = \{1\}$覆盖了所有的子句, 即$\varphi$可满足.

2.) 若$f(\left\langle \mathcal{U},\mathcal{S},k\right\rangle) = \left\langle \varphi,k\right\rangle \in$ RMS. 不妨设一个可满足的赋值为$\{x_{j_1}, \cdots, x_{j_{k''}}\} = \{1\}$且$k''\le k$. 那么由构造对应的$\mathcal{S}'' = \{ X_{j_1}, \cdots, X_{j_{k''}}\}$为大小$k'' \le k$的覆盖, 即$\left\langle \mathcal{U},\mathcal{S},k\right\rangle \in$ Set-Cover.

综上所述, RMS $\in$ NP, Set-Cover $\le_P$ RMS, 故RMS是NP-complete的.
\end{answer}

\begin{problem}{4. (Do some Calculus!)}
    For functions $g_{1}, \ldots, g_{\ell}$, we define the function $\max \left(g_{1}, \ldots, g_{\ell}\right)$ via
\begin{align*}
    \left[\max \left(g_{1}, \ldots, g_{\ell}\right)\right](x)=\max \left(g_{1}(x), \ldots, g_{\ell}(x)\right)
\end{align*}
Consider the following problem. You are given $n$ \textit{piecewise linear, continuous} functions $f_{1}, \ldots, f_{n}$ defined over the interval $[0, t]$ for some integer $t$. You are also given an integer $B$. You want to decide: do there exist $k$ of the functions $f_{i_{1}}, \ldots, f_{i_{k}}$ so that
\begin{align*}
    \int_{0}^{t}\left[\max \left(f_{i_{1}}, \ldots, f_{i_{k}}\right)\right](x)\ dx \geq B ?
\end{align*}
Prove that this problem is NP-complete.

\textbf{Note: A piecewise continuous function is a function that is continuous except at a finite number of points in its domain.}
\end{problem}

\begin{answer}
记这个问题为Calc.

(1) 首先证明Calc $\in$ NP. 输入为$n$个函数$f_1, \cdots, f_n$和整数$B,k$, 考虑证书为一组下标(不超过$k$个). 注意到连续的分段线性函数的$\max$还是一个连续的分段线性函数, 故只需要验证:
\begin{align*}
    \int_{0}^{t}\left[\max \left(f_{i_{1}}, \ldots, f_{i_{k}}\right)\right](x)\ \mathrm{d} x \geq B
\end{align*}
其中$i_1, \cdots, i_k$为证书中的下标. 验证机只需计算上述积分, 验证是否大于等于$B$, 是多项式时间的.

(2) 下面证明上一道题中的RMS $\le_P$ Calc. 设$\forall \varphi \in$ RMS, 设$\varphi$有$n$的变量$x_1,\cdots, x_n$和$m$个子句$C_1, \cdots, C_m$. 考虑如下的多项式规约:
\begin{align*}
    f_{i}(x) = \begin{cases}
        1, &\forall j\le x < j+1, \text{ if $x_i \in C_j, \forall j \in \mathbb{N}^*$}, \\
        0, &\text{Otherwise}. 
    \end{cases}
\end{align*}
那么若$\varphi$的一个可满足的赋值为$\{x_{i_1}, \cdots, x_{i_{k}}\}$, 那么:
\begin{align*}
    \left[\max(f_{i_1}, \cdots, f_{i_k})\right](x) = \begin{cases}
        1, &\forall j\le x < j+1, \text{ if $x_{i_1},\cdots, x_{i_k} \in C_j$}, \forall j \in \mathbb{N}^*, \\
        0, &\text{Otherwise}.
    \end{cases}
\end{align*}  
故令$B = m$, 那么$\varphi \in$ RMS $\iff$ $f_1,\cdots, f_n \in$ Calc.

综上所述, Calc 是NP-complete的. 证毕.
\end{answer}

\end{document}