\documentclass[11pt]{article}           
\usepackage[UTF8]{ctex}
\usepackage[a4paper]{geometry}
\geometry{left=2.0cm,right=2.0cm,top=2.5cm,bottom=2.25cm}

\usepackage{xcolor}
\usepackage{paralist}
\usepackage{enumitem}
\setenumerate[1]{itemsep=1pt,partopsep=0pt,parsep=0pt,topsep=0pt}
\setitemize[1]{itemsep=0pt,partopsep=0pt,parsep=0pt,topsep=0pt}
\usepackage{comment}
\usepackage{booktabs}
\usepackage{graphicx}
\usepackage{float}
\usepackage{sgame} % For Game Theory Matrices 
% \usepackage{diagbox} % Conflict with sgame
\usepackage{amsmath,amsfonts,graphicx,amssymb,bm,amsthm}
%\usepackage{algorithm,algorithmicx}
\usepackage{algorithm,algorithmicx}
\usepackage[noend]{algpseudocode}
\usepackage{fancyhdr}
\usepackage{tikz}
\usepackage{pgfplots}
\pgfplotsset{compat=1.18}
\usepackage{graphicx}
\usetikzlibrary{arrows,automata}
\usepackage[hidelinks]{hyperref}
\usepackage{extarrows}
\usepackage{totcount}
\setlength{\headheight}{14pt}
\setlength{\parindent}{0 in}
\setlength{\parskip}{0.5 em}
\usepackage{helvet}
\usepackage{dsfont}
% \usepackage{newtxmath}
\usepackage[labelfont=bf]{caption}
\renewcommand{\figurename}{Figure}
\usepackage{lastpage}
\usepackage{istgame}
\usepackage{tcolorbox}
% \newdateformat{mydate}{\shortmonthname[\THEMONTH]. \THEDAY \THEYEAR}

\RequirePackage{algorithm}

\makeatletter
\newenvironment{algo}
  {% \begin{breakablealgorithm}
    \begin{center}
      \refstepcounter{algorithm}% New algorithm
      \hrule height.8pt depth0pt \kern2pt% \@fs@pre for \@fs@ruled
      \parskip 0pt
      \renewcommand{\caption}[2][\relax]{% Make a new \caption
        {\raggedright\textbf{\fname@algorithm~\thealgorithm} ##2\par}%
        \ifx\relax##1\relax % #1 is \relax
          \addcontentsline{loa}{algorithm}{\protect\numberline{\thealgorithm}##2}%
        \else % #1 is not \relax
          \addcontentsline{loa}{algorithm}{\protect\numberline{\thealgorithm}##1}%
        \fi
        \kern2pt\hrule\kern2pt
     }
  }
  {% \end{breakablealgorithm}
     \kern2pt\hrule\relax% \@fs@post for \@fs@ruled
   \end{center}
  }
\makeatother


\newtheorem{theorem}{Theorem}
\newtheorem{lemma}[theorem]{Lemma}
\newtheorem{proposition}[theorem]{Proposition}
\newtheorem{claim}[theorem]{Claim}
\newtheorem{corollary}[theorem]{Corollary}
\newtheorem{definition}[theorem]{Definition}
\newtheorem*{definition*}{Definition}

\newenvironment{problem}[2][Problem]{\begin{trivlist}
    \item[\hskip \labelsep {\bfseries #1}\hskip \labelsep {\bfseries #2.}]\songti}{\hfill$\blacktriangleleft$\end{trivlist}}
\newenvironment{answer}[1][Solution]{\begin{trivlist}
    \item[\hskip \labelsep {\bfseries #1.}\hskip \labelsep]}{\hfill$\lhd$\end{trivlist}}

\newcommand\1{\mathds{1}}
% \newcommand\1{\mathbf{1}}
\newcommand\R{\mathbb{R}}
\newcommand\E{\mathbb{E}}
\newcommand\N{\mathbb{N}}
\newcommand\NN{\mathcal{N}}
\newcommand\per{\mathrm{per}}
\newcommand\PP{\mathbb{P}}
\newcommand\dd{\mathrm{d}}
\newcommand\ReLU{\mathrm{ReLU}}
\newcommand{\Exp}{\mathrm{Exp}}
\newcommand{\arrp}{\xrightarrow{P}}
\newcommand{\arrd}{\xrightarrow{d}}
\newcommand{\arras}{\xrightarrow{a.s.}}
\newcommand{\arri}{\xrightarrow{n\rightarrow\infty}}
\newcommand{\iid}{\overset{\text{i.i.d}}{\sim}}

% New math operators
\DeclareMathOperator{\sgn}{sgn}
\DeclareMathOperator{\diag}{diag}
\DeclareMathOperator{\rank}{rank}
\DeclareMathOperator{\tr}{tr}
\DeclareMathOperator{\Var}{Var}
\DeclareMathOperator{\Cov}{Cov}
\DeclareMathOperator{\Corr}{Corr}
\DeclareMathOperator{\MSE}{MSE}
\DeclareMathOperator{\Bias}{Bias}
\DeclareMathOperator*{\argmax}{argmax}
\DeclareMathOperator*{\argmin}{argmin}


\definecolor{lightgray}{gray}{0.75}


\begin{document}
\kaishu

\pagestyle{fancy}
\lhead{\CJKfamily{zhkai} 北京大学}
\chead{}
\rhead{\CJKfamily{zhkai} 2024年秋\ 信息学中的概率统计(王若松)}
\fancyfoot[R]{} 
\fancyfoot[C]{\thepage\ /\ \pageref{LastPage} \\ \textcolor{lightgray}{最后编译时间: \today}}


\begin{center}
    {\LARGE \bf Homework 7}

    {姓名:方嘉聪\ \  学号: 2200017849}            % Write down your name and ID here.
\end{center}

\begin{problem}{1}
    给定未知参数 $\theta$ 的估计量 $\hat{\theta} = \hat{\theta}(X_1, X_2, \cdots, X_n)$, 证明
    \begin{align*}
        \mathrm{MSE}(\hat{\theta}) = \Var(\hat{\theta}) + \left(\mathrm{Bias}(\hat{\theta})\right)^2 = \Var(\hat{\theta}) + \left(\theta - \E(\hat{\theta})\right)^2.
    \end{align*}
\end{problem}
\begin{answer}
我们有
\begin{align*}
    \MSE(\hat{\theta}) &= \E\left[(\hat{\theta} - \theta)^2\right] = \E\left[\hat{\theta}^2\right] - 2\theta\E[\hat{\theta}] + \theta^2 \\
    &= \E\left[\hat{\theta}^2\right] - \E[\hat{\theta}]^2 + \E[\hat{\theta}]^2 - 2\theta\E[\hat{\theta}] + \theta^2 \\
    &= \Var(\hat{\theta}) + \left(\E[\hat{\theta}] - \theta\right)^2 = \Var(\hat{\theta}) + \Bias(\hat{\theta})^2.
\end{align*}
证毕.
\end{answer}

\begin{problem}{2}
    令总体 $X$ 服从概率密度函数如下的连续分布, 其中 $\theta >0$ 为未知参数,
    \begin{align*}
        f(x) = \begin{cases}
            \theta/x^2, & x \geq \theta,\\
            0, & x < \theta.
        \end{cases} 
    \end{align*}
    给定简单随机样本 $X_1, X_2, \cdots, X_n$, 求 $\theta$ 的最大似然估计量.
\end{problem}
\begin{answer}
    极大似然函数为
    \begin{align*}
        f(\theta) = \prod_{i=1}^{n} f(x_i) = \theta^n \left(\prod_{i=1}^{n} x_i^2\right)^{-1} \cdot \1_{x_1, x_2, \cdots, x_n \ge \theta}.
    \end{align*}
    注意到当 $x_1, x_2, \cdots, x_n \ge \theta$ 时, $f(\theta)$ 为关于 $\theta$ 的单调递增函数, 故最大似然估计量为 
    \begin{align*}
        \hat{\theta}_{\mathrm{MLE}} = \min\{x_1, x_2, \cdots, x_n\}.
    \end{align*}
    证毕.
\end{answer}

\begin{problem}{3}
    令总体 $X\sim \pi(\lambda)$, 即 $X$ 服从参数为 $\lambda$ 的泊松分布, $\lambda$ 为未知参数. 给定简单随机样本 $X_1, X_2, \cdots, X_n$, 
    本题中, 我们将考虑 $p = e^{-\lambda}$ 的两个不同的估计量:
    \begin{enumerate}[label=(\arabic*)]
        \item 考虑 $p$ 的矩法估计量 $\hat{p}_1 = e^{-\bar{X}}$. 这里, $\bar{X} = \frac{1}{n}\sum_{i=1}^n X_i$ 为样本均值. 判断 $\hat{p}_1$ 是否为 $p = e^{-\lambda}$ 的最大似然估计(简要说明原因, 无需严格证明),
        判断 $\hat{p}_1$ 是否为无偏估计量, 渐进无偏估计量, 一致估计量, 并计算 $\hat{p}_1$ 的均方误差. {\kaishu 提示: 参考作业二第六题.}
        \item 令 $\hat{p}_2 = \frac{1}{n}\sum_{i=1}^n \1_{\{X_i = 0\}}$. 这里
        \begin{align*}
            \1_{\{X_i = 0\}} = \begin{cases}
                1, & X_i = 0,\\
                0, & X_i > 0.
            \end{cases}
        \end{align*}
        判断 $\hat{p}_2$ 是否为无偏估计量, 渐进无偏估计量, 一致估计量, 并计算 $\hat{p}_2$ 的均方误差.
    \end{enumerate}
\end{problem}
\begin{answer}
    \begin{enumerate}[label=(\arabic*)]
        \item 极大似然函数为
        \begin{align*}
            L(p) = \prod_{i=1}^{n} \frac{e^{-\lambda}\lambda^{x_i}}{x_i!} = p^n \frac{\lambda^{\sum_{i=1}^{n}x_i}}{\prod_{i=1}^{n}x_i!} = \frac{1}{\prod_{i=1}^{n}x_i !}\cdot p^n (-\ln p)^{\sum_{i=1}^{n}x_i}.
        \end{align*}
        对 $p$ 求导, 得到
        \begin{align*}
            L'(p) = n p^{n-1}(-\ln p)^{\sum_{i=1}^{n}x_i} - p^n \frac{\sum_{i=1}^{n}x_i}{p}(-\ln p)^{\sum_{i=1}^{n}x_i - 1} \implies p_{\mathrm{MLE}} = e^{-\bar{X}} = \hat{p}_1.
        \end{align*}
        说明 $\hat{p}_1$ 是 $p = e^{-\lambda}$ 的最大似然估计. 由作业二第六题, 对于$X\sim \pi(\lambda)$ 与 任意 $t\in \R$, 我们有
        \begin{align*}
            \E\left(e^{tX}\right) = e^{\lambda(e^t - 1)} \implies \E\left(\hat{p}_1\right) =  \left(\prod_{i=1}^{n}\E\left(e^{-X_i}\right)\right)^{1/n} = e^{\lambda(e^{-1} - 1)} \neq e^{-\lambda} = p.
        \end{align*}
        故 $\hat{p}_1$ 不是无偏估计量和渐进无偏估计量. 由大数定律知, $\bar{X} \arrp \lambda$, 故由\textbf{Lemma~\ref{lemma:1}} 有, 
        \[
        \hat{p}_1 = e^{-\bar{X}} \arrp e^{-\lambda} = p.
        \] 
        故 $\hat{p}_1$ 是一致估计量. 下面计算 $\hat{p}_1$ 的均方误差,
        \begin{align*}
            \E\left(\hat{p}_1^2\right) = \E\left(e^{-2\bar{X}}\right) =  \left(\prod_{i=1}^{n}\E\left(e^{-2X_i}\right)\right)^{1/n} = e^{\lambda(e^{-2} - 1)} 
        \end{align*}
        那么 $\hat{p}_1$ 的均方误差为
        \begin{align*}
            \MSE(\hat{p}_1) = \E\left(\hat{p}_1^2\right) + p^2 - 2p\E(\hat{p}_1) = e^{\lambda(e^{-2} - 1)} + e^{-2\lambda} - 2e^{-\lambda}e^{\lambda(e^{-1} - 1)}.
        \end{align*}
        \item 注意到
        \begin{align*}
            \E\left(\hat{p}_2\right) = \E\left[\frac{1}{n}\sum_{i=1}^{n} \1_{\{X_i = 0\}}\right] = \frac{1}{n} \sum_{i=1}^{n} \PP(X_i = 0) = e^{-\lambda} = p.
        \end{align*}
        故 $\hat{p}_2$ 是无偏估计量, 渐进无偏估计量. 那么 $\MSE(\hat{p}_2) = \Var(\hat{p}_2)$, 我们先来计算 $\E\left(\hat{p}_2^2\right)$,
        \begin{align*}
            \E\left(\hat{p}_2^2\right) &= \E\left[\frac{1}{n^2}\left(\sum_{i=1}^{n} \1_{\{X_i=1\}}\right)^2\right]  \\
            &= \frac{1}{n^2} \E\left[\sum_{i=1}^{n}\1_{\{X_i=1\}}^2 + \sum_{i\neq j} \1_{\{X_i = 0 \land X_j = 0\}}\right] \\
            &= \frac{1}{n^2} \left[\sum_{i=1}^{n}\PP(X_i = 0) + \sum_{i\neq j} \PP(X_i=0) \cdot \PP(X_j =0)\right] \\
            &= \frac{1}{n^2} \left[n e^{-\lambda} + (n^2 - n) e^{-2\lambda}\right] = e^{-2\lambda} + \frac{e^{-\lambda}}{n}(1 - e^{-\lambda}).
        \end{align*}
        那么有
        \begin{align*}
            \MSE(\hat{p}_2) = \Var(\hat{p}_2) = \E\left(\hat{p}_2^2\right) - \E\left(\hat{p}_2\right)^2 = \frac{e^{-\lambda}}{n} (1 - e^{-\lambda}) \arri 0.
        \end{align*}
        故 $\hat{p}_2$ 是一致估计量. 证毕.
    \end{enumerate}
    \begin{lemma}
        \label{lemma:1}
        设 $X_n \arrp X$, $g$ 为$\R^1$上的连续函数, 则 $g(X_n) \arrp g(X)$.
    \end{lemma}
    \begin{proof}
        对于任意的 $\delta > 0$, 存在 $M > 0$ 与 $N_1 \in \N^+$, 当 $n > N_1$ 时, 有 
        \begin{align*}
            \PP\left(|X| \ge \frac{M}{2}\right) \le \frac{\delta}{4}, \quad\PP\left(|X_n - X| \ge \frac{M}{2}\right) \le \frac{\delta}{4}.
        \end{align*}
        那么有
        \begin{align*}
            \PP(|X_n| \ge M) &= \PP\left(|X_n| \ge M, |X| < \frac{M}{2}\right) + \PP\left(|X_n| \ge M, |X| \ge \frac{M}{2}\right) \\
            &\le \PP\left(|X_n - X| \ge \frac{M}{2}\right) + \PP\left(|X|\ge \frac{M}{2}\right) < \frac{\delta}{2}.
        \end{align*}
        由于 $g(x)$ 在 $\R^1$ 上连续, 故$g(x)$ 在$[-M, M]$ 上一致连续, 即对于任意 $\varepsilon > 0$, 存在 $\eta >0$, 对于 $x_1, x_2 \in [-M, M]$, 
        若 $|x_1 - x_2| < \eta$, 则 $|g(x_1) - g(x_2)| < \varepsilon$. 由于 $X_n \arrp X$, 故存在 $N_2 \in \N^+$, 当 $n > N_2$ 时, 有
        \begin{align*}
            \PP(|X_n - X| \ge \eta) < \frac{\delta}{4}.
        \end{align*}
        于是当 $n > \max\{N_1, N_2\}$ 时, 有
        \begin{align*}
            &\PP(|g(X_n) - g(X)| \ge \varepsilon) \\
            &\le \PP\left(|g(X_n) - g(X)| \ge \varepsilon, |X_n| < M, |X| < M\right) + \PP\left(\{|X_n| \ge M\} \cup \{|X| \ge M\}\right) \\
            &\le \PP\left(|X_n - X| \ge \eta\right) + \PP\left(|X| \ge \frac{M}{2}\right) + \PP\left(|X_n| \ge M\right) \\
            &\le \frac{\delta}{4} + \frac{\delta}{4} + \frac{\delta}{2} = \delta.
        \end{align*}
        由任意性可知 $g(X_n) \arrp g(X)$. 证毕.
    \end{proof}
\end{answer}

\begin{problem}{4}
    给定样本 $X_1, X_2, \cdots, X_n \sim \mathcal{N}(\mu_1, \sigma_1^2)$, $Y_1, Y_2, \cdots, Y_m \sim \mathcal{N}(\mu_2, \sigma_2^2)$, 满足 $\{X_i\}_{i=1}^n$, $\{Y_j\}_{j=1}^m$ 相互独立.
    \begin{enumerate}[label = (\arabic*)]
        \item 令 $\bar{X} = \frac{1}{n} \sum_{i=1}^{n}X_i, \bar{Y} = \frac{1}{m} \sum_{i=1}^{m}Y_i$. 给出 $X - Y$ 服从的分布.
        \item 假定 $\sigma_1^2$ 和 $\sigma_2^2$ 均已知, 利用上一问中的结果构造枢轴量并给出 $\mu_1 - \mu_2$ 的置信水平为 $1-\alpha$ 的置信区间.
        最终结果应该依赖于 $\Phi^{-1}(1-\alpha/2)$, 其中 $\Phi(x) = \int_{-\infty}^{x} \frac{1}{\sqrt{2\pi}}e^{-t^2/2}\dd t$ 为标准正态分布的分布函数.
        \item 同样假定 $\sigma_1^2$ 和 $\sigma_2^2$ 均已知, 利用 Chernoff Bound, 给出 $\mu_1 - \mu_2$ 的置信水平为 $1-\alpha$ 的置信区间. 最终结果不应依赖于标准正态分布的分布函数.
    \end{enumerate}
\end{problem}

\begin{answer}
    \begin{enumerate}[label = (\arabic*)]
        \item $\bar{X} \sim \mathcal{N}(\mu_1, \sigma_1^2/n)$, $\bar{Y} \sim \mathcal{N}(\mu_2, \sigma_2^2/m)$, 且 $\bar{X}$ 与 $\bar{Y}$ 相互独立. 故 
        \[
            X - Y \sim \mathcal{N}(\mu_1 - \mu_2, \sigma_1^2/n + \sigma_2^2/m)
        \]
        \item 构造枢轴量
        \begin{align*}
            Z = \frac{\bar{X} - \bar{Y} - (\mu_1 - \mu_2)}{\sqrt{\sigma_1^2/n + \sigma_2^2/m}} \sim \mathcal{N}(0, 1).
        \end{align*}
        记$\sigma' = \sqrt{\sigma_1^2/n + \sigma_2^2/m}$, 那么有
        \begin{gather*}
            \PP\left[-\Phi\left(1-\frac{\alpha}{2}\right) \le Z \le \Phi\left(1-\frac{\alpha}{2}\right)\right] = 1 - \alpha. \\
            \PP\left[\bar{X} - \bar{Y} - \sigma'\cdot \Phi\left(1-\frac{\alpha}{2}\right) \le \mu_1 - \mu_2 \le \bar{X} - \bar{Y} + \sigma' \cdot \Phi\left(1-\frac{\alpha}{2}\right)\right] = 1 - \alpha.
        \end{gather*}
        故 $\mu_1 - \mu_2$ 的置信水平为 $1-\alpha$ 的置信区间为
        \begin{align*}
            \bigg[\bar{X} - \bar{Y} - \sigma' \cdot \Phi\left(1-\frac{\alpha}{2}\right), \bar{X} - \bar{Y} + \sigma' \cdot \Phi\left(1-\frac{\alpha}{2}\right)\bigg].
        \end{align*}
        \item 对于 $Z \sim \mathcal{N}(0,1)$, 有$M_Z(t) = \E(e^{tZ}) = e^{t^2/2}$. 由 Chernoff Bound, 对于任意 $t > 0$, 有
        \begin{align*}
            \PP(|Z| \ge \varepsilon) \le \inf_{t > 0} \left\{e^{-t\varepsilon}M_Z(t)\right\} = 2e^{-\varepsilon^2/2}.
        \end{align*}
        $2e^{-\varepsilon^2/2} = \alpha \implies \varepsilon = \sqrt{2\ln(2/\alpha)}$. 于是有
        \begin{gather*}
            \PP\left[-\varepsilon \le \frac{\bar{X}-\bar{Y}-(\mu_1-\mu_2)}{\sigma'} \le \varepsilon\right] \ge 1 - \alpha. \\
            \PP\left[\bar{X} - \bar{Y} - \varepsilon \cdot \sigma' \le \mu_1 - \mu_2 \le \bar{X} - \bar{Y} + \varepsilon \cdot \sigma'\right] \ge 1 - \alpha.
        \end{gather*}
        故 $\mu_1 - \mu_2$ 的置信水平为 $1-\alpha$ 的置信区间为
        \begin{align*}
            \bigg[\bar{X} - \bar{Y} - \sqrt{2\ln(2/\alpha)} \cdot \sigma', \bar{X} - \bar{Y} + \sqrt{2\ln(2/\alpha)} \cdot \sigma'\bigg].
        \end{align*}
    \end{enumerate}
    证毕.
\end{answer}

\begin{problem}{5}
    在课上, 我们考虑了下述模型: 给定 $n$ 台游戏机, 第 $i$ 台游戏机的中奖概率为 $0\le p_i \le 1$, 且 $p_i$ 均为未知参数.
    在第 $t$ 轮中, 选择一台游戏机 $1\le i\le n$, 并观测到结果 $X_{t} \sim B(1, p_i)$. 这里 $X_1, X_2, \cdots $ 相互独立.

    在课上, 我们考虑了下述均匀采样策略: 对每台游戏机进行 $N$ 次观测, 并返回样本均值最大的游戏机. 
    若取 $N = O(\ln n/\varepsilon^2)$, 则有 $\PP(p_o \ge \max p_i - \varepsilon) \ge 2/3$, 
    这里 $1\le o\le n$ 为策略返回的选择.

    本题中, 我们考虑 $n=2$ 的情况, 也即给定两台游戏机, 中奖概率分别为 $p_1$ 和 $p_2$, 且 $p_1, p_2$ 为未知参数. 令 $\Delta = |p_1 - p_2|$.
    \begin{enumerate}[label=(\arabic*)]
        \item 若 $\Delta$ 为已知参数且 $\Delta > 0$, 证明采用均匀采样策略并令 $N=O(1/\Delta^2)$, 则有 
        \begin{align*}
            \PP\left(p_o = \max\{p_1, p_2\}\right) \ge \frac{2}{3}
        \end{align*}
        这里 $o=1$ 或 $o=2$ 为策略返回的选择.
        \item \textbf{(Bonus 15\%)} 若 $\Delta$ 为未知参数且 $\Delta > 0$, 设计策略, 使得以至少 $2/3$ 的概率, 下述事件同时成立:
        \begin{itemize}
            \item $p_o = \max\{p_1, p_2\}$, 这里 $o=1$ 或 $o=2$ 为策略返回的选择.
            \item 策略的总观测次数于 $1/\Delta$ 为多项式关系.
        \end{itemize}
    \end{enumerate}
\end{problem}

\begin{answer}
    \begin{enumerate}[label=(\arabic*)]
        \item 令$\varepsilon = \Delta/2$, 由已证的结论, 对于均匀采样策略, 取$N = O(\ln 2/\varepsilon^2) = O(1/\Delta^2)$, 有
        \begin{align*}
            \PP(p_o \ge \max_{i=1,2} p_i - \varepsilon) = \PP(p_o \ge \max_{i=1,2} p_i - \Delta/2) = \PP(p_o = \max \{p_1,p_2\}) \ge \frac{2}{3}
        \end{align*}
        证毕.
        \item 考虑如下算法, 主要思想是每一轮将两个游戏机均运行若干次, 如果采样均值低于某个临界值, 则将该游戏机从备选游戏机中删去, 每一轮将临界值缩小2倍(这个算法对于$n$台游戏机的情况也成立).
        \begin{algo}
            \caption{\textbf{Successive Reject Algorithm}}
            \begin{algorithmic}[1]
                \Require 输入 $n = 2$, 待定参数 $\delta > 0$ 和 每一轮运行的次数 $T$
                \Ensure 输出 $o$
                \State $S_0 \leftarrow [n], t \leftarrow 0$
                \While{$|S_t| > 1$}
                    \State $t \leftarrow t+1$, $\varepsilon_t \leftarrow 2^{-t}$.
                    \State 对于每个 $i\in S_{t-1}$, 运行 $T$ 次.
                    \State $S_t \leftarrow \{i\in S_{t-1}: \hat{p}_{i,t} \ge \max_{j\in S_{t-1}} \hat{p}_{j,t} - \varepsilon_t\}$
                \EndWhile
                \State \Return $S_t$    
            \end{algorithmic}
        \end{algo}
        其中 $\hat{p}_{i,t}$ 为第 $t$ 轮中第 $i$ 台游戏机的样本均值. 注意到 $p_1, p_2 \in [0,1]$, 在第$t$轮, 由 Hoeffding 不等式, 有
        \begin{align*}
            \PP\left(|p_i - \hat{p}_{i,t}| > \frac{\varepsilon_t}{2}\right) \le 2e^{-2T\varepsilon_t^2}  
        \end{align*}
        取$T = \log(nt^2/\delta) / \varepsilon_t^2$, 那么有
        \begin{align*}
            \PP\left(|p_i - \hat{p}_{i,t}| > \frac{\varepsilon_t}{2}\right) \le \frac{2\delta}{nt^2}, \quad \forall i\in [n], \forall t\ge 1.
        \end{align*}
        由 Union Bound, 有
        \begin{align}
            \label{eq:1}
            \PP\left(\forall i\in [n], |p_i - \hat{p}_{i,t}| \le \frac{\varepsilon_t}{2}\right) \ge 1 - \frac{2\delta}{t^2}.
        \end{align}
        记事件(不妨设$p_1 > p_2$)
        \begin{align*}
            E_t = \{1 \in S_t \text{ and } \forall j, p_j < p_1 - 2 \varepsilon_t \Rightarrow j \notin S_t\}.
        \end{align*}
        对于 $E_0$, $\varepsilon_0 = 1$, $S_0 = [2]$, $p_2 > p_1 - 2 \varepsilon_0 = p_1 - 2$, 故$\PP(E_0) = 1$. 
        
        记事件 $E = \bigwedge_{t=1}^{\infty} E_t$. 注意到 $E$ 表示$S_t$ 中剩下的游戏机是 $1$. 故我们来分析 $\PP(E)$. 注意到\footnote{第一个不等号需要分析事件$E_t|E_{t-1}$, 证明在作业最后一页. }
        \begin{align}
            \label{eq:2}
            \PP(E_t|E_{t-1}) \ge \PP\left(\forall i\in S_{t-1}, |p_i - \hat{p}_{i,t}| \le \frac{\varepsilon}{2} \bigg| E_{t-1}\right) \ge 1 - \frac{2\delta}{t^2}.
        \end{align}
        那么有
        \begin{align*}
            \PP(E) &= \PP\left(\bigwedge_{t=1}^{\infty} E_t\right) = \prod_{t=1}^{\infty} \PP(E_t|E_{t-1}) \ge \prod_{t=1}^{\infty} \left(1 - \frac{2\delta}{t^2}\right) \ge 1 - \sum_{i=1}^{\infty} \frac{2\delta}{i^2} \ge 1 - 2\delta\cdot \frac{\pi^2}{6} 
        \end{align*}
        令$\delta = 1/\pi^2$, 那么有 $\PP(E) \ge 2/3$. 下面来证明算法的总观测次数与$1/\Delta$为多项式关系. 
        
        在 $n=2$ 的情况下, 游戏机$1$的运行次数不会超过游戏机$2$的运行次数. 而对于游戏机$2$, 由\eqref{eq:1}知, 有至少 2/3 的概率有
        \begin{align*}
            |p_2 - \hat{p}_{2,t}| \le \frac{\varepsilon_t}{2}, \quad |p_1 - \hat{p}_{1,t}| \le \frac{\varepsilon_t}{2}. \implies \hat{p}_{1,t} - \hat{p}_{2,t} - \varepsilon_t \le \Delta \le \hat{p}_{1,t} - \hat{p}_{2,t} + \varepsilon_t.
        \end{align*}
        故当 $t = \log_2(1/\Delta) + 2$ 时, 有
        \begin{align*}
            \hat{p}_{1,t} - \hat{p}_{2,t} \ge \Delta - \varepsilon_t > \varepsilon_t. \implies \hat{p}_{2,t} < \hat{p}_{1,t} - \varepsilon_t. \implies 1 \notin S_t.
        \end{align*}
        故游戏机 2 的总运行次数不超过
        \begin{align*}
            \sum_{t=1}^{\lceil\log_2(1/\Delta) + 2\rceil} \frac{\log(2t^2 / \delta)}{2^{-2t}} < \sum_{t=1}^{\lceil\log_2(4/\Delta)\rceil} \frac{\log(2\log^2(4/\Delta))}{2^{-2t}} = O\left(\frac{1}{\Delta^2}\left(\log\log^2(1/\Delta) + C\right)\right) 
        \end{align*}
        其中$C$为与$\Delta$无关的常数. 故总的观测次数为 $\mathrm{poly}(1/\Delta)$. 证毕.
    \end{enumerate}
    注: 对于\eqref{eq:2}中的第一个不等号, 我们来证明 $\{\forall i \in S_{t-1}, |p_i - \hat{p}_{i,t}| \le \varepsilon/2 | E_{t-1}\} \subseteq E_t|E_{t-1}$.
    \begin{itemize}
        \item 由于 $\forall i \in S_{t-1}, \hat{p}_{i,t}\in p_i \pm \varepsilon/2$, 有
        \begin{gather*}
            \max_{j\in S_{t-1}} \hat{p}_{j,t} - \varepsilon_t \le \max_{j\in S_{t-1}} p_j - \frac{\varepsilon}{2} < p_1 - \varepsilon  \\
            \implies \hat{p}_{1,t} \ge p_1 - \frac{\varepsilon}{2} >  \max_{j\in S_{t-1}} \hat{p}_{j,t} - \varepsilon_t. \implies 1 \in S_t.
        \end{gather*}
        \item 假设存在 $j_0$, $p_{j_0} < p_1 - 2\varepsilon$, 且 $j_0 \in S_t$. 那么有
        \begin{gather*}
            \hat{p}_{j_0,t} \ge \max_{j\in S_{t-1}} \hat{p}_{j,t} - \varepsilon_t, \quad \hat{p}_{j_0,t} \le p_{j_0} + \frac{\varepsilon}{2} < p_1 - \frac{3\varepsilon}{2}  \implies p_1 > \max_{j\in S_{t-1}} \hat{p}_{j,t} + \frac{\varepsilon}{2}
        \end{gather*}
        与 $\forall i \in S_{t-1}, |p_i - \hat{p}_{i,t}| \le \varepsilon/2$ 矛盾. 故 $\forall j, p_j < p_1 - 2 \varepsilon_t \Rightarrow j \notin S_t$.
    \end{itemize}
    综上, $\{\forall i \in S_{t-1}, |p_i - \hat{p}_{i,t}| \le \varepsilon/2 | E_{t-1}\} \subseteq E_t|E_{t-1}$. 证毕.
\end{answer}
\end{document}